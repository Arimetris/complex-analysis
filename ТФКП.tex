\documentclass[a4paper, 12pt]{article}

\usepackage{mathtools}
\usepackage{graphicx}
\usepackage[english, russian]{babel}
\usepackage[T2A]{fontenc}
\usepackage[utf8]{inputenc}
\usepackage{soulutf8}
\usepackage{nicematrix}
\usepackage{tikz}
\usetikzlibrary{angles, quotes}

\usepackage{pgfplots}

\usepackage{mathrsfs}

%настройка точек
\usepackage{tocloft}
\renewcommand{\cftsecleader}{\cftdotfill{\cftdotsep}}

\usepackage{amsfonts} %Для красивых C, R, Q
\usepackage{amsthm} %Работа с теоремами
\usepackage{amsmath} %Добавили команду text
\usepackage{indentfirst} %Делает первый абзац с отступом
\usepackage{amssymb}

\usepackage[left=25mm, right=10mm, top=20mm, bottom=20mm]{geometry}
\renewcommand{\baselinestretch}{1.25}


\renewcommand{\thesection}{\arabic{section}.} % Для разделов
\renewcommand{\thesubsection}{\thesection\arabic{subsection}.} % Для подразделов
\renewcommand{\thesubsubsection}{\thesubsection\arabic{subsubsection}.} % Для подподразделов

%Мой стиль теорем без точек
\newtheoremstyle{nodot}
{3pt}               % Пробел сверху
{3pt}               % Пробел снизу
{\itshape}          % Шрифт тела (курсив)
{0pt}               % Отступ слева (0pt — без отступа)
{\bfseries}         % Шрифт заголовка (жирный)
{}                  % Пунктуация после заголовка (пусто для nodot, или {.} для точки)
{5pt plus 1pt minus 1pt}  % Пробел после заголовка (гибкий)
{}                  % Кастомный заголовок (пусто для стандартного)

\theoremstyle{nodot}
\newtheorem*{theorem}{Теорема}
\newtheorem*{lemma}{Лемма}
\newtheorem*{definition}{Опр.\\ \indent}
\newtheorem*{statement}{Утв. \indent}
\newtheorem*{consequence}{Следствие \\}

\usepackage{titleps}
\newpagestyle{main}
{
	\setheadrule{0.4pt}
	\sethead{Комплексный анализ}{}{Лекция \arabic{section}}
	%\setfootrule{0.4pt}
	\setfoot{}{\thepage}{}
}

\newpagestyle{сontent}
{
	\setheadrule{0.4pt}
	\sethead{Комплексный анализ}{}{}
	%\setfootrule{0.4pt}
	\setfoot{}{\thepage}{}
}

\newcommand{\Cm}{\mathbb{C}}
\newcommand{\R}{\mathbb{R}}
\newcommand{\Q}{\mathbb{Q}}
\newcommand{\Z}{\mathbb{Z}}
\newcommand{\N}{\mathbb{N}}

\renewcommand{\phi}{\varphi}
\renewcommand{\epsilon}{\varepsilon}
\renewcommand{\kappa}{\varkappa}

\usepackage[
unicode=true,
colorlinks=true,      % цветные ссылки вместо рамок
urlcolor=blue,        % цвет URL-ссылок (внешних)
linkcolor=blue,       % цвет внутренних ссылок (на разделы, рисунки)
citecolor=blue,       % цвет ссылок на библиографию
filecolor=magenta]{hyperref} % альтернативно: сделать ВСЕ ссылки синими


\begin{document}
	
	\begin{titlepage}
		\begin{center}
			{\fontsize{16}{18}\selectfont ФЕДЕРАЛЬНОЕ ГОСУДАРСТВЕННОЕ АВТОНОМНОЕ ОБРАЗОВАТЕЛЬНОЕ УЧРЕЖДЕНИЕ}\\[0.4cm]
			{\fontsize{16}{18}\selectfont ВЫСШЕГО ОБРАЗОВАНИЯ}\\[0.4cm]
			{\fontsize{16}{18}\selectfont «НАЦИОНАЛЬНЫЙ ИССЛЕДОВАТЕЛЬСКИЙ УНИВЕРСИТЕТ»}\\[0.4cm]
			{\fontsize{16}{18}\selectfont «ВЫСШАЯ ШКОЛА ЭКОНОМИКИ»}\\[1cm]
			
			{\fontsize{16}{18}\selectfont Московский институт электроники и математики имени А.Н. Тихонова}\\[0.2cm]
			{\fontsize{14}{16}\selectfont\itshape Департамент прикладной математики}
			\vspace{0.5cm}
			
			\includegraphics[width=7cm]{HSE log.png}\\[1cm]
			
			{\fontsize{18}{22}\selectfont\bfseries Конспект лекций по курсу}\\[0.3cm]
			{\fontsize{20}{24}\selectfont\bfseries «Теория функций комплексного переменного»}\\[1.5cm]
			
			{\fontsize{14}{16}\selectfont \emph{Авторы конспекта:}}\\[0.2cm]
			{\fontsize{14}{16}\selectfont Падерин Артемий}\\[0.2cm]
			{\fontsize{14}{16}\selectfont Кожевников Максим}\\[3cm]
			
			{\fontsize{12}{14}\selectfont Москва, 2025 год}
		\end{center}
	\end{titlepage}
	
	\newpage
	\thispagestyle{empty}
	
	\begin{center}
		\vspace*{2cm}
		{\fontsize{16}{18}\selectfont\bfseries Благодарности}\\[1.5cm]
	\end{center}
	
	\begin{flushleft}
		\setlength{\parindent}{0pt}
		\setlength{\parskip}{1.2em}
		
		{\fontsize{14}{16}\selectfont
			Этот конспект создан на основе лекций по курсу «Теория функций комплексного переменного», читаемого в НИУ ВШЭ (МИЭМ) для студентов второго курса бакалавриата «Прикладная математика».
			
			Мы выражаем искреннюю благодарность преподавателю курса, Беловой Марии Владимировне, за интересное и доступное изложение сложного материала.
			
			Особую благодарность хотим выразить \textbf{Артемьеву Никите} за предоставленные письменные конспекты лекций, которые послужили основой для данного издания.
			
			Надеемся, что этот конспект будет полезен не только нам, но и другим студентам, изучающим комплексный анализ.
			
			Конспект оформлен в системе \LaTeX\ с использованием пакетов для математической типографики. Все иллюстрации созданы с помощью TikZ.
		}
		
		\vspace{1.5cm}
		
		\begin{flushright}
			{\fontsize{12}{14}\selectfont \emph{Авторы:}}\\[0.2cm]
			{\fontsize{12}{14}\selectfont Падерин Артемий}\\[0.1cm]
			{\fontsize{12}{14}\selectfont Кожевников Максим}\\[0.5cm]
			{\fontsize{11}{13}\selectfont Москва, 2026 год}
		\end{flushright}
	\end{flushleft}
	
	\newpage
	\pagestyle{main}
	\tableofcontents
	\newpage

	\section{Лекция 1}
	\subsection{Комплексные числа и комплексная плоскость.}
	
	\begin{definition}
		\textbf{Комплексным числом} называют выражение вида: 
		$z = a+ib, \text{ где } a,b \in \R \\ \text{и } i \text{"---} \text{мнимая единица} \ (i^2 = -1)$.
	\end{definition}
	
	\begin{flushleft}
		\begin{minipage}{0.35\textwidth}
			\begin{tikzpicture}[>=stealth]
				% Оси
				\draw[->] (-0.5,0) -- (4,0) node[below] {$x$};
				\draw[->] (0,-0.5) -- (0,3) node[left] {$y$};
				
				% Вектор
				\coordinate (O) at (0,0);
				\coordinate (A) at (3,2);
				\draw[->, thick, black] (O) -- (A) node[above right] {$z$};
				\draw[dashed] (A) -- (3,0) node[below] {$x$};
				\draw[dashed] (A) -- (0,2) node[left] {$y$};
				
				% Угол φ
				\draw (1,0) arc (0:atan2(2,3):1) node[midway, right] {$\phi$};
			\end{tikzpicture}
		\end{minipage}
		\begin{minipage}{0.3\textwidth}
			\[ 
			\begin{cases}
				x = r\cos{\phi}\\
				y = r\sin{\phi}
			\end{cases} 
			\]
		\end{minipage}
	\end{flushleft}
	
	\noindent
	$ |z| = r = \sqrt{x^2 + y^2}, \ r\geqslant0 $\\
	$ \operatorname{arg} z = \phi \text{ "--- главное значение аргумента } z,\ \phi\in(-\pi, \pi]$\\
	$ \operatorname{Arg} z = \phi +2\pi k,\ k \in \Z$ \\
	$ \operatorname{Re} z = x \text{ "--- вещественая часть }$\\
	$ \operatorname{Im} z = y \text{ "--- мнимая часть }$\\
	$ z = r (\cos\phi + i\sin\phi) \text{ "--- тригонометрическая форма записи }$\\
	$ z = re^{i\phi} \text{ "--- показательная форма записи }$
	\begin{statementname}
		\[ e^{i\phi} = \cos \phi + i\sin \phi, \ \forall \phi \in \R \]
	\end{statementname}
	
	\begin{proof}
		\[ 
			e^s = \sum\limits_{k=1}^{\infty}\frac{s^{\mathrlap{k}}}{k!}, 
			\ \ \sin s = \sum\limits_{k=1}^{\infty}\frac{(-1)^{\mathrlap{2k-1}}}{(2k-1)!},
			\ \ \cos s = \sum\limits_{k=1}^{\infty}\frac{(-1)^{\mathrlap{2k}}}{(2k)!}\\ % Делаю степень нулевой длинны
		\]
		\begin{multline*}
			e^{i\phi} = \sum\limits_{k=1}^{\infty} \frac{(i\phi)^{\mathrlap{k}}}{k!} = \{ k = 2m, \ k = 2m - 1 \} = 
			\sum\limits_{m=0}^{\infty} \frac{i^{2m}\phi^{\mathrlap{2m}}}{(2m)!} + 
			\sum\limits_{m=1}^{\infty} \frac{i^{2m-1}\phi^{\mathrlap{2m-1}}}{(2m-1)!} = \\
			= \sum\limits_{m=0}^{\infty} \frac{(-1)^{m}\phi^{\mathrlap{2m}}}{(2m)!} + 
			i\sum\limits_{m=1}^{\infty} \frac{(-1)^{2m-1}\phi^{\mathrlap{2m-1}}}{(2m-1)!} =
			\cos{\phi} + i\sin{\phi}
		\end{multline*}
	\end{proof}
	\newpage
	\begin{consequence}
		\[ \cos\phi = \frac{e^{i\phi}+e^{-i\phi}}{2}, \ \sin \phi = \frac{e^{i\phi}-e^{-i\phi}}{2i} \]
	\end{consequence}
	
	\begin{proof}
		\begin{align}
		&e^{i\phi} = \cos \phi + i\sin \phi \\
		&e^{-i\phi} = \cos \phi - i\sin \phi
		\end{align}
		\begin{flalign*}
			&\frac{(1) + (2)}{2} \Rightarrow  \cos \phi = \frac{e^{i\phi}+e^{-i\phi}}{2}, \quad
			\frac{(1) - (2)}{2} \Rightarrow  \sin \phi = \frac{e^{i\phi}-e^{-i\phi}}{2i}&
		\end{flalign*}
	\end{proof}
	
	\subsection{Операции над комплексными числами.}
	\noindent
	\begin{flalign*}
		1) \ &z_1+z_2 = (x_1+x_2) + i(y_1+y_2) & \\
		2) \ &z_1\cdot z_2 = (x_1+iy_1)(x_2+iy_2) = x_1x_2-y_1y_2 + i(x_1y_2+x_2y_1) & \\
		& z_1\cdot z_2 = r_1e^{i\phi_1}\cdot r_2e^{i\phi_2} & \\
		3) \ &\bar z = x - iy & \\
		\ & \bar{z} = \overline{re^{i\phi}} = re^{-i\phi} & \\
		4) \ &\frac{z_1}{z_2} = \frac{z_1 \bar {z_2}}{z_2 \bar {z_2}} = \frac{z_1 \bar {z_2}}{|z_2|^2}, \ z \ne 0 &\\
		\ &\frac{z_1}{z_2} = \frac{r_1}{r_2} e^{i(\phi_1 - \phi_2)} &\\
		5) \ & z_1 = z_2 \iff x_1 = x_2 \land y_1 = y_2 &\\
	\end{flalign*}
	
	\newpage
	\subsection{Расширенная комплексная плоскость.}
	\noindent
	\[ \Cm = \{ z = x+iy, \ x,y\in\R, \ i^2 = -1 \} \text{ "--- Поле комплексных чисел}\]
	\[ \overline\Cm = \Cm \cup \infty \text{ "--- Расширенная комплексная плоскость} \]
	Рассмотрим сферу: $S = \{\xi^2+\eta^2+(\zeta-\frac{1}{2})^2 = \frac{1}{4}\}$\\
	\includegraphics[width=1\textwidth]{Проекция.png}
	Из подобия в треугольнике: \[\frac{\rho}{r} = \frac{1-\zeta}{1} \iff\\  \rho = r(1-\zeta),\ SZ = r,\ AB = \rho,\ AS = \zeta,\ NS = 1 \]
	$\xi = \rho\cos\phi = r(1 - \zeta)\cos\phi = (1 - \zeta)x$\\
	$\eta = \rho\sin\phi = r(1 - \zeta)\sin\phi = (1 - \zeta)y$\\
	Подставляем в исходное уравнение:\\
	\[ (1 - \zeta)^2x^2 + (1 - \zeta)^2y^2 + (\zeta-\frac{1}{2})^2 = \frac{1}{4} \iff (1 - \zeta)^2(x^2 + y^2) + (\zeta-1)\zeta = 0 \]
	
	\[ (1 - \zeta)\bigg((1 - \zeta)(x^2+y^2)-\zeta\bigg) = 0 \]
	
	\[
	\left\{
	\begin{array}{rcl}
		\zeta &= \dfrac{x^2+y^2}{x^2+y^2+1} \\
		\xi   &= \dfrac{x}{x^2+y^2+1} \\
		\eta  &= \dfrac{y}{x^2+y^2+1}
	\end{array}
	\right.
	\iff
	\left\{
	\begin{array}{rcl}
		u &= \dfrac{x}{|z|^2+1} \\
		v &= \dfrac{y}{|z|^2+1} \\
		w &= \dfrac{|z|^2}{|z|^2+1}
	\end{array}
	\right.
	\]
	
	\noindent
	$ S \leftrightarrow \Cm$\\
	$ \Cm \mapsto S $ "--- стереографическая проекция $(\infty \mapsto N)$
	
	\begin{statementname}
		\[ \Cm \text{"--- Метрическое пространство, где } \rho(z_1, z_2) = |z_1 - z_2| \]
	\end{statementname}
	
	\begin{proof}
		 \[ |\tilde z_1-\tilde z_3| = |\tilde z_1 - \tilde z_2 + \tilde z_2 - \tilde z_3| \leqslant |\tilde z_1-\tilde z_2|+|\tilde z_2-\tilde z_3|\]
		 $\text{т.е. требуется доказать, что } |z_1+z_2 | \leqslant |z_1|+|z_2|$
		 \noindent
		\begin{flalign*}
			|z_1+z_2|^2 &=(z_1+z_2)\overline{(z_1+z_2)} = (r_1e^{i\phi_1}+r_2e^{i\phi_2})(r_1e^{-i\phi_1}+r_2e^{-i\phi_2}) =&\\
			&= r_1^2+r_2^2 + r_1r_2(e^{i(\phi_1-\phi_2)}+e^{-i(\phi_1-\phi_2)}) = r_1^2+r_2^2+2\cos(\phi_1-\phi_2)r_1r_2 = \\
			&\leqslant r_1^2+r_2^2+2r_1r_2 = (r_1+r_2)^2 = (|z_1|+|z_2|)^2
		\end{flalign*}
	\end{proof}
	\newpage
	\section{Лекция 2}
	\subsection{Примеры}
	\noindent
	\begin{flalign*}
	\textbf{1)} \quad e^{(\phi_1+\phi_2)i} &= e^{i\phi_1}\cdot e^{i\phi_2} = (\cos\phi_1 + i\sin\phi_1)(\cos\phi_2 + i\sin\phi_2) = &\\
	&= (\cos\phi_1\cos\phi_2 - \sin\phi_1\sin\phi_2) + i(\cos\phi_1\sin\phi_2 + \sin\phi_1\cos\phi_2) = &\\
	&= \cos{(\phi_1+\phi_2)} + i\sin{(\phi_1+\phi_2)}
	\end{flalign*}
	\begin{flalign*}
	\textbf{2})\quad &S_1 = 1 + \cos x + \ldots + \cos{nx} &\\
	&S_2 = \sin x + \ldots + \sin{nx} &\\
	&S_1 + iS_2 = 1 + (\cos x + i\sin x) + \ldots + (\cos {nx} + i\sin {nx}) = 1 + e^{ix} + \ldots + e^{inx} = &\\
	&= \frac{1\cdot(1-e^{i(n+1)x})}{(1-e^{ix})} = \frac{(e^{i(n+1)x}-1)}{(e^{ix}-1)}\cdot\frac{e^\frac{-ix}{2}}{e^\frac{-ix}{2}}  = \frac{(e^{i(n+1)x}-1)(e^\frac{-ix}{2})}{(e^\frac{ix}{2} - e^\frac{-ix}{2})\cdot\frac{2i}{2i}} =&\\  &=\frac{(e^{i(n+1)x}-1)(e^\frac{-ix}{2})}{2i\sin{\frac{x}{2}}}\cdot\frac{e^{-\frac{i(n+1)x}{2}}}{e^{-\frac{i(n+1)x}{2}}} 
	\frac{e^\frac{i(n+1)x}{2} - e^\frac{-i(n+1)x}{2}}{2i}\cdot\frac{1}{\sin\frac{x}{2}}\cdot\frac{e^\frac{-ix}{2}}{e^{-\frac{i(n+1)x}{2}}} = \frac{\sin\frac{(n+1)x}{2}}{\sin\frac{x}{2}}\cdot e^\frac{inx}{2}
	\end{flalign*}
	
	\[ S_1 = \operatorname{Re}S = \frac{\sin\frac{(n+1)x}{2}}{\sin\frac{x}{2}}\cdot\cos\frac{nx}{2}, \ x \neq 2\pi k \]
	\[S_2 = \operatorname{Im}S = \frac{\sin\frac{(n+1)x}{2}}{\sin\frac{x}{2}}\cdot\sin\frac{nx}{2}, \ x \neq 2\pi k \]
	
	\begin{flalign*}
	\textbf{3)}\quad &\text{Дана рекурентная последовательность: } u_{n+2} = 2(u_{n+1} - u_n), \ u_0 = u_1 = 1&\\
	&\text{Найти общий член этой последовательности.}&\\
	&\text{Замена: } u_n = \lambda^n, \quad \lambda^{n+2} = 2\lambda^{n+1} - 2\lambda^n \iff \lambda^2-2\lambda+2=0&\\
	&\lambda_{1,2} = 1\pm\sqrt{1-2} = 1 \pm i. \text{ Далее, будем искать } u_n \text{ в виде: }&\\
	&u_n = C_1(1+i)^n+C_2(1-i)^n. \text{ Найдем } C_1 \text{ и } C_2 \text{ используя начальные условия:}
	\end{flalign*}
	\[
	\left\{
	\begin{array}{rcl}
		u_0 &=& C_1 + C_2 = 1\\
		u_1 &=& C_1(1+i) + C_2(1-i) = 1
	\end{array}
	\right.
	\iff
	\left\{
	\begin{array}{rcl}
		C_1 =\frac{1}{2}\\ 
		C_2 =\frac{1}{2}
	\end{array}
	\right.
	\]
	\\
	\[ u_n = \frac{1}{2}((1+i)^n+(1-i)^n) = \frac{(\sqrt{2})^n}{2}(e^{i\frac{\pi n}{4}}+e^{-i\frac{\pi n}{4}}) = 2^\frac{n}{2}\cos\frac{\pi n}{4}\]
	\newpage
	\begin{flalign*}
		\textbf{4)}\quad &f(x) = \frac{1}{x^2+1} = \frac{A}{x-i} + \frac{B}{x+i}= \frac{i}{2}\left( \frac{1}{x+i} - \frac{1}{x-i}\right) &\\
		&f^{(k)}(x) = \frac{i(-1)^k k!}{2}\left( \frac{1}{(x+i)^{k+1}} -\frac{1}{(x-i)^{k+1}}\right) = \frac{i(-1)^k k!}{2}\left( \frac{(x-i)^{k+1}}{(x^2-1)^{k+1}} -\frac{(x+i)^{k+1}}{(x^2-1)^{k+1}}\right) &\\ \\
		&(x-i)^{k+1} - (x+i)^{k+1} = \sum\limits_{j=0}^{k+1}\binom{k+1}{j} (-i)^j x^{k+1-j} - \sum\limits_{j=0}^{k+1}\binom{k+1}{j} i^j x^{k+1-j} = \{ k = 2m\} =&\\ 
		&= 2i \sum\limits_{l=1}^{m+1}\binom{2m+1}{2l-1} (-1)^l x^{2m+1 - 2l+1}&\\
		&f^{(2m)}(x) = \frac{-(2m)!}{(x^2+1)^{2m+1}}\sum\limits_{l=1}^{m+1}\binom{2m+1}{2l-1}(-1)^lx^{2(m-l+1)}\\
		&f^{(2m+1)}(x) = \frac{(2m-1)!}{(x^2+1)^{2m}}\sum\limits_{l=1}^{m}\binom{2m}{2l-1}(-1)^lx^{2(m-l)+1}
	\end{flalign*}
	\subsection{Многочлены и формула Муавра.}
	\begin{statementname}
		(Основная теорема алгебры)\\ \indent
		Любой многочлен степени n над полем $ \Cm $ имеет n корней с учетом кратности.\\
	\end{statementname}
	\noindent
	\textbf{Формула Муавра}\\
	\indent
	Пусть $z = re^{i\phi}$ и $A = |A|e^{i(\operatorname{arg}A+2\pi k)}$, тогда\\ $z_n = \sqrt[n]{|A|}e^\frac{i(\operatorname{arg}A+2\pi k)}{n},\  k = 0,\ \ldots,\ n-1$ \\
	Корни такого многочлена распологаются в вершинах правильного n-угольника, с центром в начале координат, радиуса $\sqrt[n]{|A|}$.\\
	
	\begin{theoremname}
		(Фробениуса)\\ \indent
		Несуществует других расширений систем и чисел, кроме $\ \R\ $ и $ \ \Cm$, так чтобы новая система образовывала поле.(Нестрогая формулировка)
	\end{theoremname}
	\newpage
	\section{Лекция 3}
	\subsection{Кривые и множества.}
	\noindent
	\begin{definition}
		\hspace{-0.93cm}
		$ U_\epsilon(z_0) = \{\ z \in \Cm : |z-z_0|<\epsilon \ \} $\\
		$ \mathring U_\epsilon(z_0) = \{\ z \in \Cm : 0<|z-z_0|<\epsilon \ \} $\\
		$ U_\epsilon(\infty) = \{\ z \in \overline\Cm : |z|>\epsilon \ \} $\\
		$ \mathring U_\epsilon(\infty) = \{\ z \in \overline\Cm : \epsilon<|z|<\infty \ \} $		
	\end{definition}
	\begin{definition}
		Точка $z_0 \in \Cm$ называется\\
		1) \textbf{Внутренней} точкой множества $M\subset\Cm$, если $\ \exists U_\epsilon(z_0)\subset M$.\\
		2) \textbf {Предельной} точкой множества $M\subset\Cm$, если $\ \forall \mathring U_\epsilon(z_0)\subset M \ \Rightarrow \ \mathring U_\epsilon(z_0)\cap M \ne \varnothing$.\\
		3) \textbf {Граничной} точкой множества $M\subset\Cm$, если $\ \forall U_\epsilon(z_0) \ \exists$ точки из $M$ и не из $M$.
	\end{definition}
	\begin{definition}
		Множество вида: $\gamma =\{\ z \in \Cm : z(t) = x(t) + iy(t),\ t \in [\alpha, \beta]  \ \}$ называют \textbf{кривой} на $\Cm$\\
		$z(\alpha)$ "--- начальная точка\\
		$z(\beta)$ "--- конечная точка\\
		$t \uparrow$ "--- ориентация
	\end{definition}
	\begin{definition}
		Кривая $\gamma$ называется\\ 
		1) \textbf{Непрерывной}, если $\ x(t), y(t) \in C[\alpha, \beta]$\\
		2) \textbf{Замкнутой}, если $\ z(\alpha) = z(\beta)$\\
		3) \textbf{Гладкой}, если $\ x(t), y(t) \in C^1[\alpha, \beta] \ $ и $\ \forall t \in [\alpha, \beta] \Rightarrow z'(t) \ne 0$
	\end{definition}
	\begin{definition}
		Непрерывная кривая $\gamma$ называется \textbf{простой} или \textbf{кривой Жордана}, если $z(t_1) = z(t_2)$ выполняется только для двух точек $\alpha, \beta$.
	\end{definition}
	\begin{definition}
		Кривая $\gamma$ называется \textbf{кусочно-гладкой}, если $\gamma = \bigcup\limits_{i=1}^{N} \gamma_i$, где N "--- конечное число гладких кривых.
	\end{definition}
	\begin{definition}
		Множество $M\subset\Cm$ называют\\
		1) \textbf{Открытым}, если $\forall z_0 \in M$ "--- внутренняя точка M.\\
		2) \textbf{Связным}, если $\forall z_1, z_2 \in M \Rightarrow \exists$ Кривая Жордана $\gamma \subset M : z_1, z_2 \in \gamma $
	\end{definition}
	\begin{definition}
		\textbf{Область} "--- открытое связное множество $D \subset \Cm$.
	\end{definition}
	\begin{definition}
		Область D \textbf{односвязная}, если $\partial D$ "--- связное множество.
	\end{definition}
	\newpage
	\begin{definition}
		Область D \textbf{многосвязная}, если $\partial D$ "--- не является связным множеством и\\ $\partial D = \bigcup\limits_{i=1}^{N}\gamma, \text{ где } \gamma_1, \ldots \gamma_N$ "--- связные кривые не имеющие общих точек.
	\end{definition}
	
	\begin{center}
		\begin{tikzpicture}[scale=1]
			\draw[fill=blue!10] (0,0) circle (1);
			\node at (0,-1.5) {Односвязная};
		\end{tikzpicture}
		\hspace{2cm}
		\begin{tikzpicture}[scale=1]
			\draw[fill=blue!10] (0,0) circle (1);
			\fill[white] (0,0) circle (0.5);
			\draw (0,0) circle (0.5);
			\node at (0,-1.5) {Двусвязная};
		\end{tikzpicture}
		\hspace{2cm}
		\begin{tikzpicture}[scale=1]
			\draw[fill=blue!10] (0,0) circle (1);
			\fill[white] (-0.4,0.3) circle (0.2);
			\fill[white] (0.4,-0.3) circle (0.2);
			\draw (-0.4,0.3) circle (0.2);
			\draw (0.4,-0.3) circle (0.2);
			\node at (0,-1.5) {Трёхсвязная};
		\end{tikzpicture}
	\end{center}
	
	\subsection{Пределы последовательностей.}
	\begin{definition}
		Число $z$ называется пределом последовательности $\{z_n\}$, если
		\[ \forall\epsilon>0 \ \exists \ \delta = \delta(\epsilon)>0 : \exists N(\epsilon)\in\N : \forall n > N \Rightarrow |z_n - z| < \epsilon \ \Leftrightarrow \ \displaystyle\lim_{n \to \infty} z_n = z \]
	\end{definition}
	\begin{definition}
		Число $\infty$ называется пределом последовательности $\{z_n\}$, если
		\[ \forall\epsilon>0 \ \exists N(\epsilon)\in\N : \forall n > N \Rightarrow |z_n| > \epsilon \ \Leftrightarrow \ \displaystyle\lim_{n \to \infty} z_n = \infty \]
	\end{definition}
	\begin{theoremname}
		\[ \displaystyle\lim_{n \to \infty} z_n = z, \text{ где }z = a+ib, \ z_n = x_n+iy_n \ \Leftrightarrow \ \displaystyle\lim_{n \to \infty} x_n = a, \displaystyle\lim_{n \to \infty} y_n = b \]
	\end{theoremname}
	\begin{proof}
		\ \\$\boxed\Rightarrow$ \[ \forall\epsilon>0 \ \exists N(\epsilon)\in\N : \forall n > N \Rightarrow |z_n -z| = \sqrt{(x_n -a)^2 + (y_n - b)^2} < \epsilon \]
		\begin{flalign*}
			&|x_n -a| < \epsilon \Rightarrow \displaystyle\lim_{n \to \infty} x_n = a &\\
			&|y_n -b| < \epsilon \Rightarrow \displaystyle\lim_{n \to \infty} y_n = b &
		\end{flalign*}
		\noindent
		$\boxed\Leftarrow$ \[ \displaystyle\lim_{n \to \infty} x_n = a, \displaystyle\lim_{n \to \infty} y_n = b \]
		\begin{flalign*}
			&\forall\epsilon>0 \ \exists N_1(\epsilon)\in\N : \forall n > N_1 \Rightarrow |x_n -a| < \epsilon &\\
			&\forall\epsilon>0 \ \exists N_2(\epsilon)\in\N : \forall n > N_2 \Rightarrow |y_n -b| < \epsilon &
		\end{flalign*}
		\vspace{-1cm}
		\begin{multline*}
			\hspace{-0.5cm}
			\text{ Фиксируем } \epsilon > 0,\ N(\epsilon) = \operatorname{max}\{N_1(\epsilon),\ N_2(\epsilon)\} :\\ 
			\forall n>N(\epsilon) \ \Rightarrow |z_n - z|^2 = (x_n -a)^2 + (y_n - b)^2 < 2\epsilon^2 \Leftrightarrow |z_n - z| < \sqrt 2 \epsilon
		\end{multline*}
	\end{proof}
	\newpage
	\begin{theoremname}
		(Арифметические св-ва пределов)\\
			\indent
			$\text{Пусть } \exists\displaystyle\lim_{n \to \infty} z_n = z, \ \exists\displaystyle\lim_{n \to \infty} s_n = s, \text{тогда}$
		\vspace{-0.25cm}
		\begin{flalign*}
			&1)\ \displaystyle\lim_{n \to \infty} (z_n+s_n) = z + s&\\
			&2)\ \displaystyle\lim_{n \to \infty} (z_n\cdot s_n) = z\cdot s&\\
			&3)\ \displaystyle\lim_{n \to \infty} \left(\frac{z_n}{s_n}\right)  = \frac{z}{s},\  \forall n \in \N \ s_n \ne 0,\ s \ne 0&
		\end{flalign*}
	\end{theoremname}
	\begin{theoremname}
		(Единственность)\\
		\indent
		$\text{Если } \exists\displaystyle\lim_{n \to \infty} z_n = z, \text{тогда он единственный.}$
	\end{theoremname}
	\begin{definition}
		$\{ z_{n_k} \}_{k=1}^{+\infty} \subset \{ z_{n} \}_{k=1}^{+\infty}$ "--- подпоследоватедьность последовательности $z_n$, где\\ $n_1 < n_2 < \ldots < n_k < \ldots$
	\end{definition}
	\begin{definition}
		Последовательность  $\{ z_n \}$ называется ограниченной, если\\
		$ \exists K > 0 : \forall n \in \N \Rightarrow |z_n| \leqslant K$
	\end{definition}
	\begin{theoremname}
		(Больцано-Вейерштрасса)\\
		\indent Из всякой ограниченной последовательности можно выбрать сходящуюся подпоследовательность.
	\end{theoremname}
	\begin{theoremname}
		(Критерий Коши)
		\[ \displaystyle\lim_{n \to \infty} z_n = z \ \Leftrightarrow \ \forall\epsilon > 0 \ \exists N \in \N : \forall n, m > N \Rightarrow |z_n - z_m| < \epsilon \]
	\end{theoremname}
	\begin{example}
		\[ z_n = \sqrt[n]{n} + i(1+\frac{2}{n})^n = 1 + ie^2\]
	\end{example}
	\begin{flalign*}
		&\displaystyle\lim_{n \to \infty} \sqrt[n]{n} = \displaystyle\lim_{n \to \infty}n^\frac{1}{n} = \displaystyle\lim_{n \to \infty} e^{\frac{1}{n}\ln n} = e^{\ \displaystyle\lim_{n \to \infty} \frac{\ln n}{n}}= e^0 = 1&\\
		&\displaystyle\lim_{n \to \infty} (1 + \frac{2}{n})^n = e^{\ \displaystyle\lim_{n \to \infty}n\ln{(1 + \frac{2}{n})}} = e^{\ \displaystyle\lim_{n \to \infty} n \frac{2}{n}} = e^2&
	\end{flalign*}
	\begin{example}
		\[ z_n = \frac{\sqrt{n^2-3} - \sqrt{n^2+5}}{\sqrt{n^2+2} - n} + i(-1)^n \text{ "---расходится, т.к. } (-1)^n \text{ расходится.}\]
	\end{example}
	\newpage
	
	\section{Лекция 4}
	\subsection{Функции. Предел. Непрерывность.}
	\begin{definition}
		\textbf{ Однозначная функция } "--- отображение, ставящее в соответствие любому $z \in D$ единственное число $ w \in \overline\Cm $.\ ($ w = f(z), D_f$ "--- область определения f(z))
	\end{definition}
	\begin{definition}
		\textbf{ Многозначная функция } "--- отображение, ставящие в соответсвтие любому $z \in D$
		некоторое множество значений $ E(z) \subset \overline\Cm $.
	\end{definition}
	\begin{example}
		\[ f(z) = e^z,\ z\in\Cm,\  D_f = \Cm \] 
		$e^z = e^{x+iy} = e^x(\cos y + i\sin y)\\
		e^z = e^{z+2\pi ik}, \ T = 2\pi i$\\
		Функция однозначная
	\end{example}
	\begin{example}
		\[ f(z) = \operatorname{Ln} z, \ z = re^{i(\phi + 2\pi k)} \in \Cm,\ D_f = \Cm\setminus\{0 \} \]
		$ \operatorname{Ln} z = \ln{|z|} + i(\phi + 2\pi k), \ k \in \Z $\\
		Функция многозначная
	\end{example}
	\begin{example}
		\[ \sin z = \frac{e^{iz}- e^{-iz}}{2i}, \ \cos z = \frac{e^{iz}+ e^{-iz}}{2}, \ \sh z = \frac{e^z - e^{-iz}}{2}, \ \ch z = \frac{e^z + e^{-iz}}{2}, \ D_f \in \Cm \]
	\end{example}
	\begin{example}
		\[\text{Решить уравнение: } \cos w = \frac{e^{iw}+ e^{-iw}}{2} = z \]
	\end{example}
	\begin{proof}
		\begin{flalign*}
			&\text{Замена: } S = e^{iw}, \text{ тогда: } S + \frac{1}{S} - 2z = 0\ \Leftrightarrow \ S^2-2zS+1 = 0&\\
			&S_{1, 2} = z + \sqrt{z^2 - 1} &\\
			&\text{Обратная замена: }e^{iw} = z + \sqrt{z^2 - 1} \ \Leftrightarrow \ iw = \operatorname{Ln} (z+\sqrt{z^2 - 1})&\\
			& w = -i\operatorname{Ln} (z+\sqrt{z^2 - 1})&\\
			&\operatorname{Argcos} z = -i\operatorname{Ln}(z + \sqrt{z^2 - 1}) \text{"--- многозначная функция.}(\text{Для корня берем два значения.})\\
			&\text{Когда } \operatorname{Argcos}z \text{ принимает вещественные значения?}\\
			&\ln{|z+\sqrt{z^2-1}|} = 0 \ \Leftrightarrow \ |z+\sqrt{z^2+1}| = 1 \ \Leftrightarrow \ z+\sqrt{z^2+1} = e^{it} &\\
			&z^2+1 = (e^{it}-z)^2 = e^{2it}- 2ze^{it}+z^2 \ \Leftrightarrow \ z = \frac{e^{2it}+1}{2e^{it}} = \frac{e^{it}+e^{-it}}{2} = \cos t &\\ \\
			&\text{Ответ: } M = \{ z \in \Cm : \operatorname{Im}z = 0, \ \operatorname{Re} \in [-1, 1] \}
		\end{flalign*}
	\end{proof}
	\newpage
	\begin{example}
		\[f(z) = z^n, \ n\in\N, \ D_f = \Cm \]
	Функция однозначная
	\end{example}
	\begin{example}
		\[ f(z) = z^\frac{1}{n}, \ D_f = \Cm \]
	$ z^\frac{1}{n} = \sqrt[n]{|z|} = e^\frac{i(\operatorname{arg}z+2\pi k)}{n}, \ k = 0, 1, \ldots, n-1 $\\
	Функция n - значная
	\end{example}
	\begin{example}
		\[f(z) = z^A = e^{A\operatorname{Ln}z}, \ A \in \Cm, \ D_f = \Cm \setminus \{0\} \]
		Функция многозначная
	\end{example}
	\begin{remark}
		 	\[ \operatorname{Ln} z = \ln{|z|} + i(\phi + 2\pi k), \ \ln z = \ln{|z|} + i\phi \]
	\end{remark}
	\begin{definition}
		Пусть однозначная функция определена на области $ D, z_0 \in D $, тогда число A называют пределом $f(z)\text{ при } z \to z_0$, если:
		\[ \forall \epsilon > 0 \ \exists \delta(\epsilon) > 0 : \forall z \in \ \mathring U_\delta(z_0)\cap D \Rightarrow |f(z) - A|< \epsilon \ \Leftrightarrow \ \displaystyle\lim_{z \to z_0} z = A  \]
	\end{definition}
	\begin{theoremname}
		(Арифметические свойства)\\ \indent
		Пусть однозначная функции $ f(z), g(z)$ определены на области $ D, \ z_0 \in D \text{ и }\newline \displaystyle\lim_{z \to z_0}f(z)= A, \ \displaystyle\lim_{z \to z_0}g(z)= B$, тогда
		\begin{flalign*}
		&1)\ \displaystyle\lim_{z \to z_0}(f(z)\pm g(z))= A \pm B&\\
		&2)\ \displaystyle\lim_{z \to z_0}(f(z)\cdot g(z)) = A\cdot B&\\
		&3)\ \displaystyle\lim_{z \to z_0}\frac{f(z)}{g(z)} = \frac{A}{B}, \ B\ne 0
		\end{flalign*}
	\end{theoremname}
	\begin{definition}
		Пусть однозначная функция $ f(z) $ определена на области D. Тогда $ f(z) $ называется \textbf{ непрерывной в точке } $ z_0 \in D $, если
		\[ \displaystyle\lim_{z \to z_0}f(z) = f(z_0) \]
		\indent Функция $ f(z) $ \textbf{ непрерывна на D }, если $ f(z) $ непрерывна в каждой точке $ z_0\in D $ 		
	\end{definition}
	\begin{theorem}
		Пусть $ f(z) $ непрерывна в точке $ z_0 \in D_f, \ g(w)$ непрерывна в точке $ w_0 \in D_f$ и $w_0 = f(z_0)$. Тогда $ g(f(z))$ "--- непрерывна в точке $ z_0 $.
	\end{theorem}
	\newpage
	\subsection{Производная. Дифференцируемость. Условия Коши-Римана.}
	\begin{definition}
		Пусть функция $ f(z) $ определена на области $ D, z_0 \in D $. Тогда, если $ \exists $ конечный предел отношения приращения функции к приращению аргумета, то его называют\\ \textbf{производной функции} $ f(z) $ в точке $ z_0 $.
		\[ f'(z_0) = \displaystyle\lim_{z \to z_0}\frac{f(z) - f(z_0)}{z - z_0} \]
	\end{definition}
	\begin{definition}
		Пусть однозначная функция $ f(z) $  определена на области $D, \ z_0 \in D $. Тогда функция $ f(z) $
		называется \textbf{дифференцируемой в точке} $ z_0 $, если 
		\[ \Delta f(z_0) = A\Delta z + \alpha(z; \Delta z_0)\Delta z, \text{ где } \displaystyle\lim_{\Delta z \to 0}\alpha = 0.\]
	\end{definition}
	\begin{theorem}
		Функция $ f(z)$ дифференцируема в точке $z_0 \in D \ \Leftrightarrow \ f(z)$ имеет производную в $ z_0 $.\\ При этом $ A = f'(z), \ df(z_0) = f'(z_0)dz $ "--- главная линейная часть (дифференциал).\\
	\end{theorem}
	Далее будем рассматривать функции вида: 
	\begin{flalign*}
		&f(z) = u(x, y) + iv(x, y), \ z = x + iy, \ \operatorname{Re}f(z) = u(x, y),\ \operatorname{Im}f(z) = v(x, y)&\\
		&\Delta u(x_0, y_0) = u(x, y) - u(x_0, y_0), \ \Delta x = x - x_0, \ \Delta y = y - y_0&\\
		&\Delta u(x_0, y_0) = A\Delta x + B\Delta y + \beta(x_0, y_0, \Delta x, \Delta y)\sqrt{\Delta x^2 + \Delta y^2}, \ \displaystyle\lim_{\rho \to 0}\beta = 0.
	\end{flalign*}
	\newpage
	\begin{theorem}
		Пусть функция $f(z)$ определена на области $D$, $z_0 \in D$. Тогда следующие условия эквивалентны:
		\begin{enumerate}
			\item $f(z)$ дифференцируема в точке $z_0$.
			\item Функции $u(x,y)$ и $v(x,y)$ дифференцируемы в точке $(x_0,y_0)$ и удовлетворяют условиям Коши–Римана:
			\[
			\begin{cases}
				u'_x(x_0,y_0) = v'_y(x_0,y_0)\\
				u'_y(x_0,y_0) = -v'_x(x_0,y_0)
			\end{cases}
			\]
		\end{enumerate}
	\end{theorem}
	\begin{proof}
		\ \\$\boxed\Rightarrow$\\
		$ f(z) $ "--- дифференцируема в точке $ z_0 $, тогда 
		\[ \Delta f(z_0) = f'(z_0)\Delta z + \alpha(z_0, \Delta z)\Delta z,\ \displaystyle\lim_{\Delta z \to 0}\alpha(z_0, \Delta z) = 0 \]
		\[ f'(z_0) = a+ib, \ \alpha = \beta +i\delta \] 
		\[ \Delta z = \Delta x + i\Delta y, \ \Delta f(z_0) = \Delta u(x_0, y_0) + i\Delta v(x_0, y_0) \]
		Подставим в определение дифференцируемости:
		\[ \Delta u(x_0, y_0) + i\Delta v(x_0, y_0) = (a + ib)(\Delta x + i\Delta y) + (\beta +i\delta)(\Delta x + i\Delta y) \]
		\[
		\begin{cases}
			\Delta u(x_0, y_0) = a\Delta x - b\Delta y + \beta\Delta x - \delta\Delta y\\
			\Delta v(x_0, y_0) = a\Delta y + b\Delta x + \beta\Delta y + \delta\Delta x
		\end{cases}
		\displaystyle\lim_{(\Delta x, \Delta y) \to 0}\beta = 0, 
		\displaystyle\lim_{(\Delta x, \Delta y) \to 0}\delta = 0 \ \Rightarrow
		\]
		\[ \Rightarrow u(x, y), \ v(x, y) \text{"--- дифференцируемы в } (x_0, y_0) \text{ и } 
		\begin{cases}
		  a = u'_x(x_0, y_0), \ b = -u'_y(x_0, y_0)\\
		  a = v'_y(x_0, y_0), \ b = v'_x(x_0, y_0)
		\end{cases}
		\]
		\ \\$\boxed\Leftarrow$\\
		$ u, v $ "--- дифференцируемы в точке $ (x_0, y_0) $ и удовлетворяют условию Коши-Римана
		\[ \Delta u(x_0, y_0) = a\Delta x - b\Delta y +\beta \rho \]
		\[ \Delta v(x_0, y_0) = b\Delta x + a\Delta y +\delta \rho \]
		\[ \Delta f(z_0) = \Delta u(x_0, y_0) + \Delta v(x_0, y_0) = a(\Delta x + i\Delta y) + b(i\Delta x - i\Delta y) + (\beta + i\delta)\rho = \]
		\[ = (a+ib)(\Delta x + i\Delta y) + \frac{\alpha (|\Delta z|)}{\Delta z}\Delta z  = A\Delta z + \tilde\alpha\Delta z, \displaystyle\lim_{\Delta z \to 0}\tilde\alpha = 0 \ \rho = \sqrt{\Delta x^2 + \Delta y^2} = |\Delta z| \]
	\end{proof}
	\begin{consequencename}
		\begin{align*}
			f'(z_0) = u'_x(x_0, y_0) + iv'_x(x_0, y_0)\\
			f'(z_0) = v'_y(x_0, y_0) + iv'_x(x_0, y_0)\\
			f'(z_0) = u'_x(x_0, y_0) - iu'_y(x_0, y_0)\\
			f'(z_0) = v'_y(x_0, y_0) - iu'_y(x_0, y_0)
		\end{align*}
	\end{consequencename}
	\newpage
	
	
	\section{Лекция 5}
	\subsection{Примеры}
	
	
	\noindent
	\begin{example}
		\ \\ \indent Исследовать $f(z)$ на дифференцируемость на $\Cm$: $f(z) = z^2 - 2(\overline z)^2$
	\end{example}
	\begin{proof}
		\[ f(z) = (x + iy)^2 - 2(x - iy)^2 = x^2 - y^2 + 2xyi - 2x^2 + 2y^2 + 4xyi = y^2 - x^2 + 6xyi \]
		\[ u(x, y) = y^2 - x^2,  \quad v(x, y) = 6xy \]
		\[ u, v - \text{ дифференцируемые в } \R^2 \]
		\[ u'_x(x, y) = -2x,\ u'_y(x, y) = 2y \]
		\[ v'_x(x, y) = 6y,\ v'_y(x, y) = 6x \]
		Решим уравнения, чтобы найти точки, где выполняется К-Р:
		\[
		\begin{cases}
			u'_x(x, y) = v'_y(x, y) \implies -2x - 6x = 0 \implies x_0 = 0 \\
			u'_y(x, y) = -v'_x(x, y) \implies -2x - 6x = 0 \implies y_0 = 0
		\end{cases}
		\]
		Тогда получаем, что $f(z)$ дифференцируема только в точке $z_0 = 0$ и ее производная равна:
		\[f'(0) = u'_x(0, 0) + iv'_x(0, 0) = 0 \]
	\end{proof}
	\vspace{-0.5cm}
	\begin{example}
		\ \\ \indent Показать, что в точке $z_0 = 0$ для $f(z) = \sqrt{|xy|}$ выполнены условия К"---Р,\\ но $f(z)$ не дифференцируема в этой точке:
	\end{example}
	\begin{proof}
		\[ u = \sqrt{|xy|}, \quad v = 0 \]
		\[ v'_x = 0, \quad v'_y = 0 \]
		\[ u'_x(0, 0) = \displaystyle\lim_{x \to 0}\frac{u(x, 0) - u(0, 0)}{x - 0} = 0 \]
		\[ u'_y(0, 0) = \displaystyle\lim_{y \to 0}\frac{u(0, y) - u(0, 0)}{y - 0} = 0 \]
		Покажем, что $u(x,y)$ не дифференцируема в $(0, 0)$:
		\[ u'_x = v'_y, \quad u'_y = -v'_x, \quad (x_0, y_0) = (0, 0) \]
		Вопрос в следующем, можем ли мы представить нашу функцию как:
		\[ \Delta u(0, 0) \stackrel{?}{=} u'_x(0, 0)\Delta x + u'_y(0, 0)\Delta y + \mu \sqrt{\Delta x^2 + \Delta y^2} \text{, где } \mu \text{--- бесконечно малая}\]
		\[ \Delta u(0, 0) = \mu \sqrt{\Delta x^2 + \Delta y^2} \]
		\[\displaystyle\lim_{(\Delta x, \Delta y) \to (0, 0) }\mu = 0  \implies \displaystyle\lim_{(\Delta x, \Delta y) \to (0, 0) }\frac{\Delta u(0, 0)}{\sqrt{\Delta x^2 + \Delta y^2}} = \displaystyle\lim_{(\Delta x, \Delta y) \to (0, 0) }\frac{\sqrt{|\Delta x \Delta y|}}{\sqrt{\Delta x^2 + \Delta y^2}} = \displaystyle\lim_{( x,  y) \to (0, 0) }\frac{\sqrt{| x  y|}}{\sqrt{ x^2 +  y^2}}\]
		Нам нужно показать, что если такого предела не существует или он не равен 0.
		\newpage
		\noindent\textbf{Способ 1:}\\
		Рассмотрим этот предел вдоль прямой $y = kx$:
		\[ \displaystyle\lim_{( x,  y) \to (0, 0) }\frac{\sqrt{| x  y|}}{\sqrt{ x^2 +  y^2}} = \displaystyle\lim_{x \to 0 }\frac{\sqrt{| kx^2 |}}{\sqrt{ (1 +  k^2)x^2}} = \frac{\sqrt{| k |}}{\sqrt{ 1 +  k^2}}  \implies \nexists \displaystyle\lim_{( x,  y) \to (0, 0) }\frac{\sqrt{| x  y|}}{\sqrt{ x^2 +  y^2}}\] \\
		\textbf{Способ 2:}\\
		Перейдем в полярные координаты $x = r \cos \phi$, $y = r \sin \phi$:
		\[ \displaystyle\lim_{( x,  y) \to (0, 0) }\frac{\sqrt{| x  y|}}{\sqrt{ x^2 +  y^2}}  = \displaystyle\lim_{r \to 0}\frac{r\sqrt{| \sin\phi\cos\phi|}}{r} = \sqrt{| \sin\phi\cos\phi|} \implies \nexists \displaystyle\lim_{( x,  y) \to (0, 0) }\frac{\sqrt{| x  y|}}{\sqrt{ x^2 +  y^2}}\]
		В итоге получили, что предела не существует.
		\[\implies u(x, y) \text{ не дифференцируема в } (0, 0) \implies f(z) \text{ не дифференцируема в } z_0 = 0 \]
	\end{proof}
	\newpage
	\subsection{Условие Коши-Римана в полярной СК}
	
	\[ \text{Пусть } \ \Um (r, \phi) = u (r \cos \phi, r \sin \phi), \ \Vm (r, \phi) = v (r \cos \phi, r \sin \phi)\]
	\[f(z) = \Um(r, \phi) + i \Vm(r, \phi)\]
	\begin{statementname}
		\[\begin{cases}
			\Um'_r(r, \phi) = (\frac{1}{r})\Vm'_{\phi}(r, \phi) \\
			(\frac{1}{r})\Um'_{\phi}(r, \phi) = -\Vm'_r(r, \phi)
		\end{cases}\]
	\end{statementname}
	\begin{proof}
		\begin{align*}
			\Um'_r(r, \phi) &= u'_x x'_r + u'_y y'_r = u'_x \cos\phi + u'_y \sin\phi \\
			\Um'_\phi(r, \phi) &= u'_x x'_\phi + u'_y y'_\phi = u'_x (-r\sin\phi) + u'_y (r\cos\phi) \\
			\Vm'_r(r, \phi) &= v'_x x'_r + v'_y y'_r = v'_x \cos \phi + v'_y \sin \phi \\
			\Vm'_\phi(r, \phi) &= v'_x x'_\phi + v'_y y'_\phi = v'_x (-r\sin\phi) + v'_y (r\cos\phi)
		\end{align*}
		Знаем, что $\begin{cases}u'_x = v'_y\\ u'_y = -v'_x\end{cases}$, тогда $ \implies r\Um'_r = \Vm'_{\phi} \text{ и } -r\Vm'_r = \Um'_{\phi}$
	\end{proof}
	\subsection{Аналитические функции}
	В следующих определениях $f(z)$ --- однозначная функция.
	\begin{definition}
		Функция $f(z)$ называется \textbf{аналитической функцией} в $z_0$, если \\ $\exists \ \epsilon > 0 : f(z)$ дифференцируема в $O_{\epsilon}(z_0)$. 
	\end{definition} 
	\begin{definition}
		Функция $f(z)$ называется \textbf{аналитической в области $D$}, если \\
		она аналитична во всякой точке области D.
	\end{definition}
	\begin{remark}
		Обозначается $f(z) \in \mathcal{A}(z_0)$ и $f(z) \in \mathcal{A}(D)$ cоответственно.
	\end{remark}
	\begin{example}
		\ \\ \indent
		Выше доказали, что функция $f(z) = z^2 - 2(\overline z)^2$ дифференцируема только в точке\\ $z_0 = 0 \implies f(z) \notin \mathcal{A}(\Cm)$
	\end{example}
	\begin{example}
		\ \\ \indent Функция $f(z) = e^z \implies f(z) \in \mathcal{A}(\Cm)$
	\end{example}
	\begin{example}
		\ \\ \indent Функция $f(z) = \frac{1}{z - 1} \implies f(z) \in \mathcal{A}(\Cm \setminus \{1\})$
	\end{example}
	\newpage
	
	\section{Лекция 6}
	\subsection{Геометрический смысл модуля и аргумента производной}
	Возьмем функцию $f(z) \in \mathcal{A}(D), z_0 \in D \text{, где } f'(z_0) \neq 0$, а так же рассмотрим гладкую кривую Жордана:
	\[ \gamma_z = \{ z \in \Cm : z = z(t), \alpha \leq t \leq \beta \}\]
	Построим отображение нашей кривой из плоскости $z$ в плоскость $\omega$:
	\begin{center}
		\begin{tikzpicture}[
		>=Stealth,
		curve/.style={very thick, blue!60!black},
		region/.style={fill=blue!10, draw=black, thin},
		point/.style={circle, fill=red, inner sep=2pt, outer sep=0pt},
		tangent/.style={thick, red!70!black},
		label/.style={font=\small, black},
		angle/.style={draw=green!50!black, thick},
		plane/.style={circle, draw=black, thick, inner sep=2pt} % стиль для кружков плоскостей
		]
		
		% ===============================
		% Область D с кривой \gamma_z
		% ===============================
		\begin{scope}[local bounding box=scope1]
			% Область D
			\fill[region] (0.5,0) to[out=30, in=210] (2,1.5)
			to[out=30, in=150] (5,0.5)
			to[out=330, in=90] (4,-1.5)
			to[out=270, in=30] (0.3,-1)
			to[out=210, in=330] cycle;
			
			% Точка z_0 - ПОДПИСАНА КРАСИВО
			\coordinate (z0) at (2.3, 0.77);
			\node[point, label={[label] above right:$z_0$}] at (z0) {};
			% Дополнительная подпись точки (по желанию)
			\node[above left=0.03cm, font=\bfseries\tiny, black!100] at (z0) {$z_0$};
			
			% Кривая \gamma_z
			\draw[curve] (1,-0.5) to[out=40, in=200] 
			(2,0.3) to[out=20, in=180] (z0)
			to[out=0, in=160] (3.5,0.6) 
			to[out=340, in=140] (4,0);
			
			% Касательная в точке z_0
			\draw[tangent] (1.85,0.6) -- (3.5,1.2);
			
			% Угол между касательной и горизонталью
			\draw[angle] (z0) ++(0.5,0) arc (0:20:0.5)
			node[midway, right, font=\tiny] {$\beta$};
			
			% Горизонтальная линия для отсчета угла
			\draw[gray, thin, dashed] (z0) -- ++(1,0);
			
			% Плоскость z в кружочке
			\node[plane, label={[label] below:$z$}] at (0.5, 1.5) {$z$};
			
			% Подпись кривой
			\node[label] at (4.3, 0.2) {$\gamma_z$};
		\end{scope}
		
		% ===============================
		% Область D' с кривой \gamma_{\omega}
		% ===============================
		\begin{scope}[local bounding box=scope2, xshift=9cm]
			% Область D'
			\fill[region, fill=green!10] (0.5,0) to[out=50, in=230] (1.7,1.5)
			to[out=50, in=130] (4.5,0.8)
			to[out=310, in=80] (4,-1.2)
			to[out=260, in=10] (1.5,-1)
			to[out=190, in=330] cycle;
			
			% Точка \omega_0 - ПОДПИСАНА КРАСИВО
			\coordinate (w0) at (2.3, 1);
			\node[point, label={[label] above right:$\omega_0$}] at (w0) {};
			% Дополнительная подпись точки
			\node[above left=0.03cm, font=\tiny, black!100] at (w0) {$w_0$};
			
			% Кривая \gamma_{\omega}
			\draw[curve] (1.2,0) to[out=50, in=210] 
			(2.2,0.5) to[out=30, in=170] (w0)
			to[out=10, in=150] (3.4,0.6)
			to[out=330, in=130] (4.2,0.3);
			
			% Касательная в точке \omega_0
			\draw[tangent] (1.75,0.7) -- (3.8,1.8);
			
			% Угол
			\draw[angle] (w0) ++(0.5,0) arc (0:28:0.5)
			node[midway, right, font=\tiny] {$\theta$};
			
			% Горизонтальная линия для отсчета угла
			\draw[gray, thin, dashed] (w0) -- ++(1,0);
			
			% Плоскость ω в кружочке
			\node[plane, label={[label] below:$\omega$}] at (1, 1.5) {$\omega$};
			
			% Подпись кривой
			\node[label] at (4, 0.7) {$\gamma_{\omega}$};
		\end{scope}
		
		% ===============================
		% ДУГОВАЯ СТРЕЛКА ОТОБРАЖЕНИЯ
		% ===============================
		% Координаты для дуговой стрелки
		\coordinate (startArrow) at ([yshift=1cm]scope1.east);
		\coordinate (endArrow) at ([yshift=0.3cm]scope2.west);
		\coordinate (midArrow) at ($(startArrow)!0.5!(endArrow) + (0,1.5cm)$);
		
		% Рисуем дуговую стрелку
		\draw[->, ultra thick, red!60!black, 
		line width=1.2pt,
		shorten >=2pt, shorten <=-13pt] 
		(startArrow) .. controls ($(startArrow)!0.4!(midArrow)$) and ($(endArrow)!0.4!(midArrow)$) .. (endArrow)
		node[midway, above, label, yshift=5pt] {$f$};
	\end{tikzpicture}\\
	\end{center}
	Тогда, $\omega_0 = f(z_0)$. Рассмотрим сложную функцию $\omega(t) = f(z(t))$ и найдем ее производную в точке $t_0$:
	\[\omega'(t_0) = f'(z_0)z'(t_0)\]
	Откуда по свойству аргумента комплексного числа и $\begin{cases} \arg (\omega'(t_0)) = \theta \\ \arg (f'(t_0)) = \alpha \\ \arg (z'(t_0)) = \beta \end{cases}$ получаем, что:
	\[\arg (\omega'(t_0)) = \arg (f'(z_0)) + \arg (z'(t_0)) \implies \theta = \alpha + \beta\]
	Тогда $\alpha = \theta - \beta$ - угол поворота кривой в точке $z_0$
	\begin{statementname}
		Геометрический смысл аргумента производной\\
		Угол между жордановыми кривыми в точке $z_0$ равен углу между их образами в точке $\omega_0$
	\end{statementname}
	\vspace{0.5cm}
	Рассмотрим выражение $|\underbrace{f'(z_0)}_{\mathclap{K \neq 0}}| = \displaystyle\lim_{\Delta t \to 0}\left|\frac{\Delta \omega}{\Delta t}\right|$, откуда следует его представление в виде:
	\[|\Delta \omega | = K|\Delta z| + \overline{\overline{o}}(|\Delta t|)\]
	Из этого следует, что бесконечно малые фигуры преобразуются в подобные им, при этом число $K$ называется \textbf{коэффициентом подобия}.
	\vspace{0.5cm}
	\begin{statementname}
		Геометрический смысл модуля производной\\
		При отображении кривых сохраняется коэффициент подобия: $|\Delta \omega| \approx k |\Delta z|$ 
	\end{statementname}
	\newpage
	\subsection{Связанные теоремы}
	В последующих теоремах считаем $f(z)$ - однозначной функцией, $D$ - область на которой она определена.
	
	\begin{theorem}
		Пусть $f(z) \in \mathcal{A}(D), \quad f(z) \cancel{\equiv} \mathrm{const}, \quad t \in D$, тогда $D_\omega = f(D_z)$ - область и любая кривая Жордана $\gamma_z \in D_z$ $\xlongrightarrow{f}$ в кривую Жордана $\gamma_\omega \in D_\omega$
	\end{theorem}
	
	\begin{theorem}
		Пусть $f(z) \in \mathcal{A}(O_{\epsilon}(z))$ и $f(z_0) \neq 0$, тогда $\exists \delta > 0$ в $O_{\delta}(\omega_0), \omega_0 = f(z_0)$ определена обратная для $\psi(\omega) \in \mathcal{A}(O_{\delta}(\omega_0)): \implies \psi'(\omega_0) = \frac{1}{f'(z_0)}$
	\end{theorem}
	
	\begin{statement}
		Пусть $f(z) \in \mathcal{A}(D_z): f: D_z \longrightarrow D_{\omega}$ гладкая кривая Жордана $\gamma_z \in D_z \longrightarrow \gamma_{\omega} \in D_{\omega}$
	\end{statement}
	
	\[\gamma_z = \{ z \in \mathbb{C}: z = \phi(t), \ \alpha \leq t \leq \beta\}\]
	
	Предположим существование этих величин:
	\[S_{D_\omega} = \underset{D_z}{\iint}|f'(x)|^2 \mathrm{d}x \mathrm{d}y \text{ --- площадь}\]
	\[L_{\gamma_{\omega}} = \int_{\alpha}^{\beta} |f'(\psi(t))|\cdot |\psi'(t)| \mathrm{d}t \text{ --- длина}\]
	
	\begin{proof}
		\[1) \quad S_{D_\omega} = \underset{D_\omega}{\iint} \mathrm{d}u \mathrm{d}v = \left|(u, v) \xrightarrow{\text{замена}} (x, y)\right| = \underset{D_z}{\iint}|\mathcal{J}(x, y)|\mathrm{d}x \mathrm{d}y = \underset{D_z}{\iint}\begin{vmatrix}
			u_x & u_y \\
			v_x & v_y
		\end{vmatrix}\mathrm{d}x \mathrm{d}y\]
		
		\[\det\mathcal{J}(x, y) = \begin{vmatrix}
			u_x & u_y \\
			v_x & v_y
		\end{vmatrix} = v_y u_x - u_y v_x = u_x^2 + v_x^2 = |f'(z)|^2 \implies S_{D_\omega} = \underset{D_z}{\iint} |f'(z)|^2 \mathrm{d}x \mathrm{d}y\]
		
		\[2) \quad L_{\gamma_z} = \underset{\gamma_z}{\int}|\mathrm{d}\omega|; \quad \omega = f(\psi(t))\]
		
		\[\mathrm{d}\omega = f'(\psi(t))\psi'(t) \mathrm{d}t \implies |\mathrm{d}\omega| = |f'(\psi(t))||\psi'(t)|\mathrm{d}t \implies L_{\gamma_{\omega}} = \int_{\alpha}^{\beta} |f'(\psi(t))|\cdot |\psi'(t)| \mathrm{d}t\]
	\end{proof}
	
	\newpage
	
	\subsection{Примеры}
	
	\begin{example}
		Пусть дана функция $f(z) = e^z$, дана область определения:
		\[ D_z = \{ z \in \mathbb{C} : 1 \leq \operatorname{Re} z \leq 2, \ 0 \leq \operatorname{Im} z \leq 4\}\]
		и кривая Жордана:
		\[\gamma_z = \{ z \in \mathbb{C}: \operatorname{Re} z = \operatorname{Im} z, \ 0 \leq \operatorname{Re} z \leq 2\pi \}\]
		Найти площадь образа области и длину образа этой кривой.
	\end{example}
	
	\begin{proof}
		\[\text{Наше отображение переводит } f: \begin{cases} D_z \longrightarrow D_\omega \\ \gamma_z \longrightarrow \gamma_\omega \end{cases}\]
		
		\begin{center}
			\begin{tikzpicture}[scale=0.8]
				\begin{scope}[xshift=0cm]
					\draw[->, thick] (-0.5,0) -- (5,0) node[right] {$\operatorname{Re} z$};
					\draw[->, thick] (0,-0.5) -- (0,5) node[above] {$\operatorname{Im} z$};
					\fill[blue!10, draw=blue, thick] (1,0) rectangle (2,4);
					\draw[dashed] (1,0) -- (1,-0.2) node[below] {$1$};
					\draw[dashed] (2,0) -- (2,-0.2) node[below] {$2$};
					\draw[dashed] (0,4) -- (-0.2,4) node[left] {$4$};
					\node[blue] at (2.4, 4.2) {$D_z$};
				\end{scope}
				
				\begin{scope}[xshift=8cm]
					\draw[->, thick] (-0.5,0) -- (5,0) node[right] {$\operatorname{Re} z$};
					\draw[->, thick] (0,-0.5) -- (0,5) node[above] {$\operatorname{Im} z$};
					\draw[red, thick] (0,0) -- (4,4);
					\node[red] at (2.7, 3.2) {$\gamma_z$};
					\draw[dashed] (4,0) -- (4,-0.2) node[below] {$2\pi$};
					\draw[dashed] (0,4) -- (-0.2,4) node[left] {$2\pi$};
				\end{scope}
			\end{tikzpicture}
		\end{center}
		
		\[f(z) = e^z \implies  f'(z) = e^z = e^{x+iy} \implies |f'(z)| = e^x\]
		
		\[ S_{D_\omega} = \underset{D_z}{\iint} |f'(z)|^2 \mathrm{d}x \mathrm{d}y = \int_0^4 \int_1^2 e^{2x} \mathrm{d}x \mathrm{d}y = 4 \cdot \left.\frac{e^{2x}}{2}\right|_1^2 = 2 (e^4 - e^2)\]
		
		Для поиска длины кривой параметризуем ее параметром $x$:
		\[\text{Из рисунка } y = x \implies z = x + ix, \ 0 \leq x < 2\pi\]
		\[|f'(z(x))| = e^x, \quad |z'(x)| = |1 + i| = \sqrt{2}\]
		
		Тогда получаем, что:
		\[L_{\gamma_{\omega}} = \int_{0}^{2\pi} |f'(\psi(x))|\cdot |\psi'(x)| \mathrm{d}x = \int_{0}^{2\pi} \sqrt{2} e^x \mathrm{d}x = \sqrt{2} \left. e^x \right |_0^{2\pi} = \sqrt{2}(e^{2\pi} - 1)\]
	\end{proof}
	
	\newpage
	
	\newpage
	\section{Лекция 7}
	\subsection{Гармонические функции}
	\begin{definition}
		Вещественнозначная функция $u(x, y)$ называется \textbf{гармонической} в области $D$, если $u(x, y) \in C^2(D)$ и удовлетворяет уравнению Лапласа: $\Delta u = 0$, где $\Delta u = u''_{xx} + u''_{yy}$, $(x, y) \in D$.
	\end{definition}
	
	\begin{remark}
		Уравнение в частных производных вида $\Delta u = 0$ называется \textbf{уравнением Лапласа}.
	\end{remark}
	
	\begin{theorem}
		Если $f(z) \in \mathcal{A}(D) \implies \operatorname{Re} f, \operatorname{Im} f$ --- гармонические функции в $D$.
	\end{theorem}
	
	\begin{proof}
		$f(z) = u(x,y) + i v(x,y)$, где $u = \operatorname{Re} f$, $v = \operatorname{Im} f$.
		
		Из условий Коши-Римана: $u'_x = v'_y$, $u'_y = -v'_x$.
		
		Дифференцируем первое уравнение по $x$, второе по $y$:
		\[u''_{xx} = v''_{yx}, \quad u''_{yy} = -v''_{xy}\]
		
		Складывая и учитывая равенство смешанных производных $v''_{xy} = v''_{yx}$, получаем:
		\[u''_{xx} + u''_{yy} = v''_{yx} - v''_{xy} = 0\]
		Аналогично для $v(x,y)$.
	\end{proof}
	
	\begin{definition}
		Две гармонические в области $D$ функции, связанные условием Коши-Римана называют \textbf{гармонически сопряженными функциями}.
	\end{definition}
	
	\begin{remark}
		Вещественные и мнимые части аналитической функции являются гармонически сопряженными функциями.
	\end{remark}
	
	Есть некоторая гармоническая функция, как построить аналитическую функцию, для которой она ее вещественная или мнимая часть?
	
	\begin{theorem}
		Пусть $u(x,y)$ гармоническая в односвязной области $D$ функция, тогда $\exists$ однозначная функция $f(z)$, такая что $f(z) \in \mathcal{A}(D)$ и $u(x,y) = \operatorname{Re} f(z)$, причем $f(z)$ определяется с точностью до аддитивной мнимой постоянной.
	\end{theorem}
	
	\begin{proof}
		Построим гармонически-сопряженную функцию. Введем обозначения $u'_x = v'_y, \quad u'_y = -v'_x$.
		
		\[dv = v'_x dx + v'_y dy = -u'_y dx + u'_x dy\]
		
		\[(*) \quad v(x,y) = \int_{(x_0, y_0)}^{(x,y)} -u'_y dx + u'_x dy + c, \quad c \in \R\]
		
		Проверим независимость интеграла от пути. Пусть $P = -u'_y$, $Q = u'_x$. Условием является равенство перекрестных производных:
		\[P_y = Q_x \implies -u''_{yy} = u''_{xx} \implies \Delta u = 0\]
		
		Как видим, пришли к уравнению Лапласа, которое выполняется по условию.
		
		Тогда аналитическая функция $f(z)$ запишется как:
		\[f(z) = u + iv = u + i \cdot\int_{(x_0, y_0)}^{(x,y)} -u'_y dx + u'_x dy + ic, \quad c \in \R\]
		Откуда видим, что наша функция определяется чисто мнимой постоянной.
	\end{proof}
	
	\begin{remark}
		Если $D$ - многосвязная область, то интеграл $(*)$ может быть многозначной функцией.
	\end{remark}

	\subsection{Примеры}
	Напоминание:
	\[u(x, y) \in C^2(D), \Delta u = 0 \text{ --- гармоническая функция}\]
	\[ f(z) = u(x, y) + i v(x, y), f \in \mathcal{A}(D) \quad u, v \text{ --- гармонически сопряженные функции}\]
	\[\begin{cases} u'_x = v'_y \\ u'_y = - v'_x \end{cases} \quad (x,y) \in D\]
	Известно $u$. Требуется найти гармонически сопряженную функцию $v$. 
	\[\begin{cases}dv = v'_x dx + v'_y dy \\ dv = -u'_y dx + u'_x dy \end{cases} \implies v(x, y) = \int _{(x_0, y_0)}^{(x, y)} (-u'_y dx + u'_x dy)\]
	
	\begin{example}
		Пусть $u = e^{-y}\cos x$, $z = x + iy \in \Cm$
		Найти $f(z)$, такое что $Re f = u$
	\end{example}
	
	\begin{proof}
		\[u'_x = -e^{-y} \sin x\]
		\[u'_y = -e^{-y} \cos x\]
		\[\begin{cases}
			v'_x = -u'_y = e^{-y} \cos x \\ 
			v'_y = u'_x = -e^{-y} \sin x
		\end{cases}\]
		
		Найдем $v$ из первого уравнения:
		\[v = e^{-y} \int \cos x dx = e^{-y} (\sin x + A(y))\]
		где $A(y)$ - произвольная функция от переменной $y$.
		
		Тогда:
		\[v = e^{-y} \sin x + B(y), \text{ где } B(y) = e^{-y}A(y)\]
		
		Найдем производную по $y$:
		\[v'_y = -e^{-y} \sin x + B'(y)\]
		
		Подставляем во второе уравнение системы:
		\[-e^{-y} \sin x + B'(y) = -e^{-y} \sin x \implies B'(y) = 0 \implies B(y) = c_0\]
		
		Таким образом:
		\[v = e^{-y} \sin x + c_0, c_0 \in \R\]
		
		Имеем:
		\[f(z) = e^{-y} \cos x + i e^{-y} \sin x + i c_0\]
		
		Воспользуемся формулами Эйлера:
		\[\sin x = \frac{e^{ix} - e^{-ix}}{2i}\]
		\[\cos x = \frac{e^{ix} + e^{-ix}}{2}\]
		
		Тогда:
		\[f(z) = \frac{e^{-y}}{2}(e^{ix} + e^{-ix}) + i \cdot \frac{e^{-y}}{2i}(e^{ix} - e^{-ix}) + i c_0\]
		\[= \frac{e^{-y}}{2}(e^{ix} + e^{-ix} + e^{ix} - e^{-ix}) + i c_0\]
		\[= e^{-y} e^{ix} + i c_0 = e^{ix - y} + i c_0\]
		\[= e^{i(x + iy)} + i c_0 = e^{iz} + i c_0\]
		
		Ответ: $f(z) = e^{iz} + i c_0, c_0 \in \R$
	\end{proof} 
	
	В общем случае не так легко получить результат, чтобы он зависел от $z$, а не от $x$ и $y$. Для этого используем:
	\[\begin{cases} z = x + iy \\ \overline{z} = x - iy\end{cases} \implies \begin{cases} x = \frac{z + \overline{z}}{2} \\ y = \frac{z - \overline{z}}{2i} \end{cases}\]
	
	\begin{example}
		Существуют ли гармонические функции вида $u(x,y) = \psi(\frac{y}{x})$? Если да, то найти их.
		\[\Delta u = 0 \implies u''_{xx} + u''_{yy} = 0\]
		\[s = \frac{y}{x}, u = \psi(s) \implies u'_y = \psi'(s) \cdot s'_y = \psi '(s) \cdot \frac{1}{x} \implies u''_{yy} = \frac{1}{x} \cdot \psi''(s) \cdot s'_y = \frac{\psi ''(s)}{x^2}\]
		\[u''_{xx} = \psi ''(s) \cdot (-\frac{y}{x^2})^2 + \frac{2y\psi'(s)}{x^3}\]
		Далее подставляем эти формулы в уравнение (*):
		\[\psi'' \cdot \frac{y^2}{x^4} + \frac{2y}{x^3}\psi' + \frac{\psi''}{x^2} = 0\]
		\[\psi'' \cdot \frac{y^2}{x^2} + \frac{2y}{x}\psi' + \psi'' = 0\]
		Получили линейное дифференциальное уравнение с переменными коэффициентами:
		\[\psi '' (s^2 + 1) + 2s \psi' = 0 \text{--- ОДУ}\]
		Сделаем замену $\psi'(s) = h(s)$
		\[h'(s^2 + 1) + 2sh = 0 \implies \frac{dh}{h} = -\frac{2s}{s^2 + 1}ds\]
		\[\ln|h| = - \ln|s^2 + 1| + \ln c_1 \implies h = \frac{c_2}{s^2 + 1} \implies \psi ' = \frac{c_2}{s^2 + 1}\]
		Решаем ОДУ первого порядка:
		\[\psi = c_2 \int \frac{ds}{s^2 + 1} \implies \psi = \arctan{s} \cdot c_2 + c_3\]
		Ответ: $u(x, y) = c_2 \cdot \arctan(\frac{y}{x}) + c_3, \quad c_2, c_3 \in \R$
	\end{example}
	
	\begin{theorem}
		Пусть $f(z) \in \mathcal{A}(D)$ и $\forall z \in D \implies f'(z) \neq 0$.
		Тогда $\{ u(x, y) = c_1\} \perp \{v(x,y) = c_2\}$, где $u = Re f, v = Im f$
	\end{theorem}
	
	\begin{proof}
		\[f(z) \in \mathcal{A}(D) \implies u, v - \text{гармонически сопряженные в D функции}\]
		\[\implies u'_x = v'_y, \quad u'_y = -v'_x, (x, y) \in D\] 
		\[\{u = c_1\} \perp \{v = c_2\} \iff \nabla u \perp \nabla v\]
		\[\nabla u = u'_x \overline{e}_x + u'_y \overline{e}_y\]
		\[\nabla v = v'_x \overline{e}_x + v'_y \overline{e}_y\]
		\[(\nabla u, \nabla v) = u'_x v'_x + u'_y v'_y = -u'_x u'_y + u'_y u'_x = 0\]
	\end{proof}
	
	\begin{remark}
		Если $\exists z_0 \in D: f'(z_0) = 0 \implies f'(z_0) = u'_x(x_0, y_0) + i v'_x(x_0, y_0) = 0$
		\[\begin{cases} u'_x (x_0, y_0) = 0 \\ v'_x(x_0, y_0) = 0 \end{cases} \implies \begin{cases} v'_y (x_0, y_0) = 0 \\ u'_y(x_0, y_0) = 0\end{cases} \implies \begin{cases} \nabla u(x_0, y_0) = \overline{0} \\ \nabla v(x_0, y_0) = \overline{0}\end{cases}\] 
	\end{remark}
	\newpage
	\subsection{Конформные отображения}
	\begin{center}
	\begin{tikzpicture}[
		>=Stealth,
		curve/.style={very thick, blue!60!black},
		region/.style={fill=blue!10, draw=black, thin},
		point/.style={circle, fill=red, inner sep=2pt, outer sep=0pt},
		tangent/.style={thick, red!70!black},
		label/.style={font=\small, black},
		angle/.style={draw=green!50!black, thick},
		plane/.style={circle, draw=black, thick, inner sep=2pt} % стиль для кружков плоскостей
		]
		
		% ===============================
		% Область D с кривой \gamma_z
		% ===============================
		\begin{scope}[local bounding box=scope1]
			% Область D
			\fill[region] (0.5,0) to[out=30, in=210] (2,1.5)
			to[out=30, in=150] (5,0.5)
			to[out=330, in=90] (4,-1.5)
			to[out=270, in=30] (0.3,-1)
			to[out=210, in=330] cycle;
			
			% Плоскость z в кружочке
			\node[plane, label={[label] below:$z$}] at (0.5, 1.5) {$z$};
			
		\end{scope}
		
		% ===============================
		% Область D' с кривой \gamma_{\omega}
		% ===============================
		\begin{scope}[local bounding box=scope2, xshift=9cm]
			% Область D'
			\fill[region, fill=green!10] (0.5,0) to [out=50, in=230] (1.7,1.5)
			to[out=50, in=130] (4.7,1)
			to[out=310, in=80] (4,-1.4)
			to[out=260, in=10] (1.2, -1.7)
			to[out=190, in=330] cycle;
			
			% Плоскость ω в кружочке
			\node[plane, label={[label] below:$\omega$}] at (1, 1.5) {$\omega$};
			
		\end{scope}
		
		% ===============================
		% ДУГОВАЯ СТРЕЛКА ОТОБРАЖЕНИЯ
		% ===============================
		% Координаты для дуговой стрелки
		\coordinate (startArrow) at ([yshift=0.4cm]scope1.east);
		\coordinate (endArrow) at ([yshift=0.1cm]scope2.west);
		\coordinate (midArrow) at ($(startArrow)!0.5!(endArrow) + (0, -1.5cm)$);
		
		% Рисуем дуговую стрелку
		\draw[->, ultra thick, red!60!black, 
		line width=1.2pt,
		shorten >=2pt, shorten <=-13pt] 
		(startArrow) .. controls ($(startArrow)!0.4!(midArrow)$) and ($(endArrow)!0.4!(midArrow)$) .. (endArrow)
		node[midway, above, label, yshift=5pt] {$f$};
	\end{tikzpicture}\\
	\end{center}
	Классы задач:\\
	I) Найти область $D_\omega$ куда после отображения отображение $f(z)$ области $D_z$ $\quad$ Дано: $D_z, \omega = f(z)$\\
	II) Найти такое отображение $\omega = f(z)$ Дано $D_z, D_\omega$
	
	\begin{definition}
		Пусть функция $f(z)$ определена на области $D_z$. Тогда, если отображение $\omega = f(z)$ взаимно-однозначно на $D_z$, то это отображение называют \textbf{однолистным}.
		(Взаимно-однозначное) = (Однолистность)
		\[\omega = f(z)\]
		\[D_z : z_1 \neq z_2 \iff D_\omega: \omega_1 \neq \omega_2\]
	\end{definition}
	
	\begin{definition}
		Пусть функция $f(z)$ определена на области $D_z$. Отображение $\omega = f(z)$ называют \textbf{конформным} на $D_z$, если: 
		\begin{enumerate} 
			\item это отображение однолистно и в обе стороны непрерывно на $D$ (\textbf{гомеоморфизм}) 
			\item это отображение в каждой точке области $D$ обладает свойством сохранения углов и постоянством коэффициента подобия.
		\end{enumerate}
	\end{definition}
	
	\begin{theorem}
		Пусть $f(z) \in \mathcal{A}(D), z_0 \in D_z$ и $f'(z_0) \neq 0$. Тогда $\exists \ \mathcal{U}_\epsilon(z_0) \subset D_z$: отображение $\omega = f(z)$ конформно на $\mathcal{U}_\epsilon(z_0)$.
	\end{theorem}
	
	\begin{proof}
		Теорема об аналитичности обратной функции $\implies$ \textbf{1)} \\ 
		Из геометрического смысла модуля и аргумента производной $\implies$ \textbf{2)}
	\end{proof}
	
	\begin{theorem} 
		Если функция $f(z) \in \mathcal{A}(D)$ и положим отображение $\omega = f(z)$ однолистно на $D_z$. Тогда отображение $\omega = f(z)$ конформно на $D_z$.
	\end{theorem}	
	Бесконечность: \\ 
	1) $z_0 \xrightarrow{f} \infty \quad (\frac{1}{\omega} = \frac{1}{f(z)})$
	\begin{definition}
		Будем говорить, что отображение $\omega = f(z)$ конформно переводит $\mathcal{U}_\epsilon(z_0)$ на $\mathcal{U}_\delta(\infty)$, если отображение $\xi = \frac{1}{f(z)}$ конформно переводит $\mathcal{U}_\epsilon(z_0)$ на $\mathcal{U}_\mu(0)$
	\end{definition}
	2) $\infty \xrightarrow{f} \omega_0 \quad (\eta = \frac{1}{z})$
	\begin{definition}
		Будем говорить, что отображение $\omega = f(z)$ конформно переводит $\mathcal{U}_\epsilon(\infty)$ на $\mathcal{U}_\delta(\omega_0)$, если отображение $\omega = f(\frac{1}{z})$ конформно переводит $\mathcal{U}_r(0)$ на $\mathcal{U}_delta(\omega_0)$
	\end{definition}
	3) $\infty \xrightarrow{f} \infty \quad (\eta = \frac{1}{z}, \xi = \frac{1}{\omega})$
	\begin{definition}
		Будем говорить, что отображение $\omega = f(z)$ конформно переводит $\mathcal{U}_\epsilon(\infty)$ на $\mathcal{U}_\delta(\infty)$, если отображение $\frac{1}{f(\frac{1}{z})}$ конформно переводит $\mathcal{U}_\mu (0)$ на $\mathcal{U}_\mathcal{M} (0)$
	\end{definition}
\end{document}