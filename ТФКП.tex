\documentclass[a4paper, 12pt]{article}

\usepackage{mathtools}
\usepackage{graphicx}
\usepackage[english, russian]{babel}
\usepackage[T2A]{fontenc}
\usepackage[utf8]{inputenc}
\usepackage{soulutf8}
\usepackage{nicematrix}
\usepackage{tikz}
\usetikzlibrary{angles, quotes}

\usepackage{pgfplots}

\usepackage{mathrsfs}

%настройка точек
\usepackage{tocloft}
\renewcommand{\cftsecleader}{\cftdotfill{\cftdotsep}}

\usepackage{amsfonts} %Для красивых C, R, Q
\usepackage{amsthm} %Работа с теоремами
\usepackage{amsmath} %Добавили команду text
\usepackage{indentfirst} %Делает первый абзац с отступом
\usepackage{amssymb}

\usepackage[left=25mm, right=10mm, top=20mm, bottom=20mm]{geometry}
\renewcommand{\baselinestretch}{1.25}


\renewcommand{\thesection}{\arabic{section}.} % Для разделов
\renewcommand{\thesubsection}{\thesection\arabic{subsection}.} % Для подразделов
\renewcommand{\thesubsubsection}{\thesubsection\arabic{subsubsection}.} % Для подподразделов

%Мой стиль теорем без точек
\newtheoremstyle{nodot}
{3pt}               % Пробел сверху
{3pt}               % Пробел снизу
{\itshape}          % Шрифт тела (курсив)
{0pt}               % Отступ слева (0pt — без отступа)
{\bfseries}         % Шрифт заголовка (жирный)
{}                  % Пунктуация после заголовка (пусто для nodot, или {.} для точки)
{5pt plus 1pt minus 1pt}  % Пробел после заголовка (гибкий)
{}                  % Кастомный заголовок (пусто для стандартного)

\theoremstyle{nodot}
\newtheorem*{theorem}{Теорема}
\newtheorem*{lemma}{Лемма}
\newtheorem*{definition}{Опр.\\ \indent}
\newtheorem*{statement}{Утв. \indent}
\newtheorem*{consequence}{Следствие \\}

\usepackage{titleps}
\newpagestyle{main}
{
	\setheadrule{0.4pt}
	\sethead{Комплексный анализ}{}{Лекция \arabic{section}}
	%\setfootrule{0.4pt}
	\setfoot{}{\thepage}{}
}

\newpagestyle{сontent}
{
	\setheadrule{0.4pt}
	\sethead{Комплексный анализ}{}{}
	%\setfootrule{0.4pt}
	\setfoot{}{\thepage}{}
}

\newcommand{\Cm}{\mathbb{C}}
\newcommand{\R}{\mathbb{R}}
\newcommand{\Q}{\mathbb{Q}}
\newcommand{\Z}{\mathbb{Z}}
\newcommand{\N}{\mathbb{N}}

\renewcommand{\phi}{\varphi}
\renewcommand{\epsilon}{\varepsilon}
\renewcommand{\kappa}{\varkappa}

\usepackage[
unicode=true,
colorlinks=true,      % цветные ссылки вместо рамок
urlcolor=blue,        % цвет URL-ссылок (внешних)
linkcolor=blue,       % цвет внутренних ссылок (на разделы, рисунки)
citecolor=blue,       % цвет ссылок на библиографию
filecolor=magenta]{hyperref} % альтернативно: сделать ВСЕ ссылки синими


\author{Падерин А.Ю.}
\date{3 Модуль}
\title{Комплексный анализ}

\begin{document}
	\maketitle %Генерация заголовка
	\thispagestyle{empty} %делаем страницу без колонтитулов
	\newpage
	
	\pagestyle{main}
	\tableofcontents
	\thispagestyle{сontent}
	\newpage
	
	\section{Лекция 1}
	\subsection{Комплексные числа и комплексная плоскость}
	
	\begin{definition}
		Комплексным числом называют выражение вида: 
		\[ z = a+ib, \text{ где } a,b \in \R \ \text{и } i \text{"---} \text{мнимая единица} \ (i^2 = -1) \]
	\end{definition}
	
	\begin{flushleft}
		\begin{minipage}{0.35\textwidth}
			\begin{tikzpicture}[>=stealth]
				% Оси
				\draw[->] (-0.5,0) -- (4,0) node[below] {$x$};
				\draw[->] (0,-0.5) -- (0,3) node[left] {$y$};
				
				% Вектор
				\coordinate (O) at (0,0);
				\coordinate (A) at (3,2);
				\draw[->, thick, black] (O) -- (A) node[above right] {$z$};
				\draw[dashed] (A) -- (3,0) node[below] {$x$};
				\draw[dashed] (A) -- (0,2) node[left] {$y$};
				
				% Угол φ
				\draw (1,0) arc (0:atan2(2,3):1) node[midway, right] {$\phi$};
			\end{tikzpicture}
		\end{minipage}
		\begin{minipage}{0.3\textwidth}
			\[ 
			\begin{cases}
				x = r\cos{\phi}\\
				y = r\sin{\phi}
			\end{cases} 
			\]
		\end{minipage}
	\end{flushleft}
	
	\noindent
	$ |z| = r = \sqrt{x^2 + y^2}, \ r\geqslant0 $\\
	$ \operatorname{arg} z = \phi \text{ "--- главное значение аргумента } z,\ \phi\in(-\pi, \pi]$\\
	$ \operatorname{Arg} z = \phi +2\pi k,\ k \in \Z$ \\
	$ \operatorname{Re} z = x \text{ "--- вещественая часть }$\\
	$ \operatorname{Im} z = y \text{ "--- мнимая часть }$\\
	$ z = r (\cos\phi + i\sin\phi) \text{ "--- тригонометрическая форма записи }$\\
	$ z = re^{i\phi} \text{ "--- показательная форма записи }$\\
	\begin{statement}
		\[ e^{i\phi} = \cos \phi + i\sin \phi, \ \forall \phi \in \R \]
	\end{statement}
	
	\begin{proof}
		\[ 
			e^s = \sum\limits_{k=1}^{\infty}\frac{s^{\mathrlap{k}}}{k!}, 
			\ \ \sin s = \sum\limits_{k=1}^{\infty}\frac{(-1)^{\mathrlap{2k-1}}}{(2k-1)!},
			\ \ \cos s = \sum\limits_{k=1}^{\infty}\frac{(-1)^{\mathrlap{2k}}}{(2k)!}\\ % Делаю степень нулевой длинны
		\]
		
		\begin{multline*}
			e^{i\phi} = \sum\limits_{k=1}^{\infty} \frac{(i\phi)^{\mathrlap{k}}}{k!} = \{ k = 2m, \ k = 2m - 1 \} = 
			\sum\limits_{m=0}^{\infty} \frac{i^{2m}\phi^{\mathrlap{2m}}}{(2m)!} + 
			\sum\limits_{m=1}^{\infty} \frac{i^{2m-1}\phi^{\mathrlap{2m-1}}}{(2m-1)!} = \\
			= \sum\limits_{m=0}^{\infty} \frac{(-1)^{m}\phi^{\mathrlap{2m}}}{(2m)!} + 
			i\sum\limits_{m=1}^{\infty} \frac{(-1)^{2m-1}\phi^{\mathrlap{2m-1}}}{(2m-1)!} =
			\cos{\phi} + i\sin{\phi}
		\end{multline*}
		
	\end{proof}
	
	\newpage
	\begin{consequence}
		\[ 
			\cos \phi = \frac{e^{i\phi}+e^{-i\phi}}{2}, \ \sin \phi = \frac{e^{i\phi}-e^{-i\phi}}{2i}
		\]
	\end{consequence}
	
	\begin{proof}
		\begin{align}
		&e^{i\phi} = \cos \phi + i\sin \phi \\
		&e^{-i\phi} = \cos \phi - i\sin \phi
		\end{align}
		
		\begin{flalign*}
			&\frac{(1) + (2)}{2} \Rightarrow  \cos \phi = \frac{e^{i\phi}+e^{-i\phi}}{2}, \quad
			\frac{(1) - (2)}{2} \Rightarrow  \sin \phi = \frac{e^{i\phi}-e^{-i\phi}}{2i}&
		\end{flalign*}
	\end{proof}
	
	\subsection{Операции над комплексными числами}
	\noindent
	\begin{flalign*}
		1) \ &z_1+z_2 = (x_1+x_2) + i(y_1+y_2) & \\
		2) \ &z_1\cdot z_2 = (x_1+iy_1)(x_2+iy_2) = x_1x_2-y_1y_2 + i(x_1y_2+x_2y_1) & \\
		& z_1\cdot z_2 = r_1e^{i\phi_1}\cdot r_2e^{i\phi_2} & \\
		3) \ &\bar z = x - iy & \\
		\ & \bar{z} = \overline{re^{i\phi}} = re^{-i\phi} & \\
		4) \ &\frac{z_1}{z_2} = \frac{z_1 \bar {z_2}}{z_2 \bar {z_2}} = \frac{z_1 \bar {z_2}}{|z_2|^2}, \ z \ne 0 &\\
		\ &\frac{z_1}{z_2} = \frac{r_1}{r_2} e^{i(\phi_1 - \phi_2)} &\\
		5) \ & z_1 = z_2 \iff x_1 = x_2 \land y_1 = y_2 &\\
	\end{flalign*}
	
	\newpage
	\subsection{Расширенная комплексная плоскость}
	\noindent
	\[ \Cm = \{ z = x+iy, \ x,y\in\R, \ i^2 = -1 \} \text{ "--- Поле комплексных чисел}\]
	\[ \overline\Cm = \Cm \cup \infty \text{ "--- Расширенная комплексная плоскость} \]
	Рассмотрим сферу: $S = \{\xi^2+\eta^2+(\zeta-\frac{1}{2})^2 = \frac{1}{4}\}$\\
	\includegraphics[width=1\textwidth]{Проекция.png}
	Из подобия в треугольнике: \[\frac{\rho}{r} = \frac{1-\zeta}{1} \iff\\  \rho = r(1-\zeta) \  ,SZ = r, AB = \rho, AS = \zeta, NS = 1 \]
	$\xi = \rho\cos\phi = r(1 - \zeta)\cos\phi = (1 - \zeta)x$\\
	$\eta = \rho\sin\phi = r(1 - \zeta)\sin\phi = (1 - \zeta)y$\\
	Подставляем в исходное уравнение:\\
	\[ (1 - \zeta)^2x^2 + (1 - \zeta)^2y^2 + (\zeta-\frac{1}{2})^2 = \frac{1}{4} \iff (1 - \zeta)^2(x^2 + y^2) + (\zeta-1)\zeta = 0 \]
	
	\[ (1 - \zeta)\bigg((1 - \zeta)(x^2+y^2)-\zeta\bigg) = 0 \]
	
	\[
	\left\{
	\begin{array}{rcl}
		\zeta &= \dfrac{x^2+y^2}{x^2+y^2+1} \\
		\xi   &= \dfrac{x}{x^2+y^2+1} \\
		\eta  &= \dfrac{y}{x^2+y^2+1}
	\end{array}
	\right.
	\iff
	\left\{
	\begin{array}{rcl}
		u &= \dfrac{x}{|z|^2+1} \\
		v &= \dfrac{y}{|z|^2+1} \\
		w &= \dfrac{|z|^2}{|z|^2+1}
	\end{array}
	\right.
	\]
	
	\noindent
	$ S \leftrightarrow \Cm$\\
	$ \Cm \mapsto S $ "--- стереографическая проекция $(\infty \mapsto N)$
	
	\begin{statement}
		\[ \Cm \text{"--- Метрическое пространство, где } \rho(z_1, z_2) = |z_1 - z_2| \]
	\end{statement}
	
	\begin{proof}
		 \[ |\tilde z_1-\tilde z_3| = |\tilde z_1 - \tilde z_2 + \tilde z_2 - \tilde z_3| \leqslant |\tilde z_1-\tilde z_2|+|\tilde z_2-\tilde z_3|\]
		 $\text{т.е. требуется доказать, что } |z_1+z_2 | \leqslant |z_1|+|z_2|$
		 \noindent
		\begin{flalign*}
			|z_1+z_2|^2 &=(z_1+z_2)\overline{(z_1+z_2)} = (r_1e^{i\phi_1}+r_2e^{i\phi_2})(r_1e^{-i\phi_1}+r_2e^{-i\phi_2}) =&\\
			&= r_1^2+r_2^2 + r_1r_2(e^{i(\phi_1-\phi_2)}+e^{-i(\phi_1-\phi_2)}) = r_1^2+r_2^2+2\cos(\phi_1-\phi_2)r_1r_2 = \\
			&\leqslant r_1^2+r_2^2+2r_1r_2 = (r_1+r_2)^2 = (|z_1|+|z_2|)^2
		\end{flalign*}
	\end{proof}
	\newpage
	\section{Лекция 2}
	\subsection{Примеры}
	\noindent
	\begin{flalign*}
	\textbf{1)} \quad e^{\phi_1+\phi_2} &= e^{i\phi_1}\cdot e^{i\phi_2} = (\cos\phi_1 + \sin\phi_1)(\cos\phi_2 + \sin\phi_2) = &\\
	&= (\cos\phi_1\cos\phi_2 - \sin\phi_1\sin\phi_2) + i(\cos\phi_1\sin\phi_2 + \sin\phi_1\cos\phi_2) = &\\
	&= \cos{(\phi_1+\phi_2)} + i\sin{(\phi_1+\phi_2)}
	\end{flalign*}
	\begin{flalign*}
	\textbf{2})\quad &S_1 = 1 + \cos x + \ldots + \cos{nx} &\\
	&S_2 = \sin x + \ldots + \sin{nx} &\\
	&S_1 + iS_2 = 1 + (\cos x + i\sin x) + \ldots + (\cos {nx} + i\sin {nx}) = 1 + e^{ix} + \ldots + e^{inx} = &\\
	&= \frac{1\cdot(1-e^{i(n+1)x})}{(1-e^{ix})} = \frac{(e^{i(n+1)x}-1)}{(e^{ix}-1)}\cdot\frac{e^\frac{-ix}{2}}{e^\frac{-ix}{2}}  = \frac{(e^{i(n+1)x}-1)(e^\frac{-ix}{2})}{(e^\frac{ix}{2} - e^\frac{-ix}{2})\cdot\frac{2i}{2i}} =&\\  &=\frac{(e^{i(n+1)x}-1)(e^\frac{-ix}{2})}{2i\sin{\frac{x}{2}}}\cdot\frac{e^{-\frac{i(n+1)x}{2}}}{e^{-\frac{i(n+1)x}{2}}} 
	\frac{e^\frac{i(n+1)x}{2} - e^\frac{-i(n+1)x}{2}}{2i}\cdot\frac{1}{\sin\frac{x}{2}}\cdot\frac{e^\frac{-ix}{2}}{e^{-\frac{i(n+1)x}{2}}} = \frac{\sin\frac{(n+1)x}{2}}{\sin\frac{x}{2}}\cdot e^\frac{inx}{2}
	\end{flalign*}
	
	\[ S_1 = \operatorname{Re}S = \frac{\sin\frac{(n+1)x}{2}}{\sin\frac{x}{2}}\cdot\cos\frac{nx}{2}, \ x \neq 2\pi k \]
	\[S_2 = \operatorname{Im}S = \frac{\sin\frac{(n+1)x}{2}}{\sin\frac{x}{2}}\cdot\sin\frac{nx}{2}, \ x \neq 2\pi k \]
	
	\begin{flalign*}
	\textbf{3)}\quad &\text{Дана рекурентная последовательность: } u_{n+2} = 2(u_{n+1} - u_n), \ u_0 = u_1 = 1&\\
	&\text{Найти общий член этой последовательности.}&\\
	&\text{Замена: } u_n = \lambda^n, \quad \lambda^{n+2} = 2\lambda^{n+1} - 2\lambda^n \iff \lambda^2-2\lambda+2=0&\\
	&\lambda_{1,2} = 1\pm\sqrt{1-2} = 1 \pm i. \text{ Далее, будем искать } u_n \text{ в виде: }&\\
	&u_n = C_1(1+i)^n+C_2(1-i)^n. \text{ Найдем } C_1 \text{ и } C_2 \text{ используя начальные условия:}
	\end{flalign*}
	\[
	\left\{
	\begin{array}{rcl}
		u_0 &=& C_1 + C_2 = 1\\
		u_1 &=& C_1(1+i) + C_2(1-i) = 1
	\end{array}
	\right.
	\iff
	\left\{
	\begin{array}{rcl}
		C_1 =\frac{1}{2}\\ 
		C_2 =\frac{1}{2}
	\end{array}
	\right.
	\]
	\\
	\[ u_n = \frac{1}{2}((1+i)^n+(1-i)^n) = \frac{(\sqrt{2})^n}{2}(e^{i\frac{\pi n}{4}}+e^{-i\frac{\pi n}{4}}) = 2^\frac{n}{2}\cos\frac{\pi n}{4}\]
	\newpage
	\begin{flalign*}
		\textbf{4)}\quad 
	\end{flalign*}
	\subsection{Многочлены и формулы муавра}
	\begin{statement1}
		Основная теорема алгебры
		\[\forall  P(z) = c_nz^n+c_{n-1}z^{n-1}+\ldots+c_0, \ c_i \in \Cm,\ c_n\ne0, \ n\in\N, \text{ имеет n корней с учетом кратности}\]
	\end{statement1}
	\noindent
	\textbf{Формула Муавра}\\
	\indent
	Пусть $z = re^{i\phi}$ и $A = |A|e^{i(\operatorname{arg}A+2\pi k)}$, тогда\\ $z_n = \sqrt[n]{|A|}e^\frac{i(\operatorname{arg}A+2\pi k)}{n},\  k = 0,\ \ldots,\ n-1$ \\
	Корни такого многочлена распологаются в вершинах правильного n-угольника, с центром в начале координат, радиуса $\sqrt[n]{|A|}$\\
	
	\begin{theorem}
		(Фробениуса)\\
		\indent
		Несуществует других расширений сисем и чисел, кроме $\ \R\ $ и $ \ \Cm$, так чтобы новая система образовывала поле.(Нестрогая формулировка)
	\end{theorem}
	\newpage
	\section{Лекция 3}
	\subsection{Кривые и множества}
	
	\begin{definition}
		Точка $z_0 \in \Cm$ называется\\
		1) \textbf{Внутренней} точкой множества $M\subset\Cm$, если $\ \exists U_\epsilon(z_0)\subset M$.\\
		2) \textbf {Предельной} точкой множества $M\subset\Cm$, если $\ \forall \mathring U_\epsilon(z_0)\subset M \ \Rightarrow \ \mathring U_\epsilon(z_0)\cap M \ne \varnothing$.\\
		3) \textbf {Граничной} точкой множества $M\subset\Cm$, если $\ \forall U_\epsilon(z_0) \ \exists$ точки, как из $M$, так и не из $M$.\\ \\
		$ U_\epsilon(z_0) = \{\ z \in \Cm : |z-z_0|<\epsilon \ \} $\\
		$ \mathring U_\epsilon(z_0) = \{\ z \in \Cm : 0<|z-z_0|<\epsilon \ \} $\\
		$ U_\epsilon(\infty) = \{\ z \in \overline\Cm : |z|>\epsilon \ \} $\\
		$ \mathring U_\epsilon(\infty) = \{\ z \in \overline\Cm : \epsilon<|z|<\infty \ \} $\\
	\end{definition}
	\begin{definition}
		Множество вида: $\gamma =\{\ z \in \Cm : z(t) = x(t) + iy(t),\ t \in [\alpha, \beta]  \ \}$ называют \textbf{кривой} на $\Cm$\\
		$z(\alpha)$ "--- начальная точка\\
		$z(\beta)$ "--- конечная точка\\
		$t \uparrow$ "--- ориентация\\
	\end{definition}
	\begin{definition}
		Кривая $\gamma$ называется\\ 
		1) \textbf{Непрерывной}, если $\ x(t), y(t) \in C[\alpha, \beta]$\\
		2) \textbf{Замкнутой}, если $\ z(\alpha) = z(\beta)$\\
		3) \textbf{Гладкой}, если $\ x(t), y(t) \in C^1[\alpha, \beta] \ $ и $\ \forall t \in [\alpha, \beta] \Rightarrow z'(t) \ne 0$\\
	\end{definition}
	\begin{definition}
		Непрерывная кривая $\gamma$ называется \textbf{простой} или \textbf{кривой Жордана}, если $z(t_1) = z(t_2)$ выполняется только для двух точек $\alpha, \beta$.\\
	\end{definition}
	\begin{definition}
		Кривая $\gamma$ называется \textbf{кусочно-гладкой}, если $\gamma = \bigcup\limits_{i=1}^{N} \gamma_i$, где N "--- конечное число гладких кривых.\\
	\end{definition}
	\begin{definition}
		Множество $M\subset\Cm$ называют\\
		1) \textbf{Открытым}, если $\forall z_0 \in M$ "--- внутренняя точка M.\\
		2) \textbf{Связным}, если $\forall z_1, z_2 \in M \Rightarrow \exists$ Кривая Жордана $\gamma \subset M : z_1, z_2 \in \gamma $\\
	\end{definition}
	\begin{definition}
		\textbf{Область} "--- открытое связное множество $D \subset \Cm$
	\end{definition}
	\begin{definition}
		\textbf{Область} "--- открытое связное множество $D \subset \Cm \ 26 $
	\end{definition}
	
\end{document}