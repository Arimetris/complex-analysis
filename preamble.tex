\documentclass[a4paper, 12pt]{article}

\usepackage{mathtools}
\usepackage{graphicx}
\usepackage[english, russian]{babel}
\usepackage[T2A]{fontenc}
\usepackage[utf8]{inputenc}
\usepackage{soulutf8}
\usepackage{nicematrix}
\usepackage{tikz}
\usetikzlibrary{angles, quotes}

\usepackage{pgfplots}

\usepackage{mathrsfs}

%настройка точек
\usepackage{tocloft}
\renewcommand{\cftsecleader}{\cftdotfill{\cftdotsep}}

\usepackage{amsfonts} %Для красивых C, R, Q
\usepackage{amsthm} %Работа с теоремами
\usepackage{amsmath} %Добавили команду text
\usepackage{indentfirst} %Делает первый абзац с отступом
\usepackage{amssymb}

\usepackage[left=25mm, right=10mm, top=20mm, bottom=20mm]{geometry}
\renewcommand{\baselinestretch}{1.25}


\renewcommand{\thesection}{\arabic{section}.} % Для разделов
\renewcommand{\thesubsection}{\thesection\arabic{subsection}.} % Для подразделов
\renewcommand{\thesubsubsection}{\thesubsection\arabic{subsubsection}.} % Для подподразделов

%Мой стиль теорем без точек
\newtheoremstyle{nodot}
{3pt}               % Пробел сверху
{3pt}               % Пробел снизу
{\itshape}          % Шрифт тела (курсив)
{0pt}               % Отступ слева (0pt — без отступа)
{\bfseries}         % Шрифт заголовка (жирный)
{}                  % Пунктуация после заголовка (пусто для nodot, или {.} для точки)
{5pt plus 1pt minus 1pt}  % Пробел после заголовка (гибкий)
{}                  % Кастомный заголовок (пусто для стандартного)

\theoremstyle{nodot}
\newtheorem*{theorem}{Теорема}
\newtheorem*{lemma}{Лемма}
\newtheorem*{definition}{Опр.\\ \indent}
\newtheorem*{statement}{Утв. \indent}
\newtheorem*{consequence}{Следствие \\}

\usepackage{titleps}
\newpagestyle{main}
{
	\setheadrule{0.4pt}
	\sethead{Комплексный анализ}{}{Лекция \arabic{section}}
	%\setfootrule{0.4pt}
	\setfoot{}{\thepage}{}
}

\newpagestyle{сontent}
{
	\setheadrule{0.4pt}
	\sethead{Комплексный анализ}{}{}
	%\setfootrule{0.4pt}
	\setfoot{}{\thepage}{}
}

\newcommand{\Cm}{\mathbb{C}}
\newcommand{\R}{\mathbb{R}}
\newcommand{\Q}{\mathbb{Q}}
\newcommand{\Z}{\mathbb{Z}}
\newcommand{\N}{\mathbb{N}}

\renewcommand{\phi}{\varphi}
\renewcommand{\epsilon}{\varepsilon}
\renewcommand{\kappa}{\varkappa}

\usepackage[
unicode=true,
colorlinks=true,      % цветные ссылки вместо рамок
urlcolor=blue,        % цвет URL-ссылок (внешних)
linkcolor=blue,       % цвет внутренних ссылок (на разделы, рисунки)
citecolor=blue,       % цвет ссылок на библиографию
filecolor=magenta]{hyperref} % альтернативно: сделать ВСЕ ссылки синими
